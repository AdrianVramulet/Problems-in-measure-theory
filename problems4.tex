\begin{problem}
    Let $(\Omega,\cF)$ be a measure space. Let $A\subseteq \Omega$ and consider the indicator function $\indE{A} : \Omega\to \bbR$, defined by 
    \begin{equation*}
        \indE{A}(\omega) = 1, \text{ if $\omega\in A$},\quad  \indE{A}(\omega) = 0, \text{ if $\omega\notin A$}.
    \end{equation*}
    Show that $\indE{A}$ is $(\cF,\cB)$-measurable if and only if $A\in \cF$.
\end{problem}

\begin{problem}
    Let $(\Omega,\cF)$ be a measure space and $A,B\in \cF$. Show that
    \begin{equation*}
        \indE{A\cup B} = \indE{A} + \indE{B} - \indE{A\cap B},\qquad \indE{A\cap B} = \indE{A} \indE{B}. 
    \end{equation*}
\end{problem}

\begin{problem}
    Let $(\Omega,\cF,\bbP)$ be a probability space, and $X,Y:\Omega\to \bbR$ be random variables. Recall that $X$ and $Y$ are called independent if
    \begin{equation*}
        \bbP(A\cap B) = \bbP(A)\bbP(B),\qquad\forall A\in \sigma(X),\, \forall B\in \sigma(Y).
    \end{equation*} 
    Show that $X$ and $Y$ are independent if and only if 
    \begin{equation*}
        \bbP(X\geq s, Y\geq t) = \bbP(X\geq s) \bbP(Y\geq t),\qquad\forall s,t\in \bbR.
    \end{equation*} 
    (Hint: you might need the $\pi-\lambda$ Theorem.)
\end{problem}

\begin{problem}
    Let $(\Omega,\cF)$ be a measurable space and $f,g:\Omega\to \bbR$ be  $\cF$-measurable functions. Show that $\max(f,g)$, $\min(f,g)$ and $|f|$ are measurable. 
\end{problem}

\begin{problem} Let $(\Omega,\cF)$ be a measurable space and $E_1,\ldots,E_n\in \cF$ be measurable sets. Show that if $a_1,\ldots,a_n$ are real numbers, then
\begin{equation*}
    f = \sum_{i=1}^n a_i \indE{E_i}
\end{equation*}
is $\cF$-measurable.
\end{problem}

\begin{problem}
    Let $(\Omega,\cF)$ be a measurable space, $f:\Omega\to \overline{\bbR}$, and $Y = f^{-1}(\bbR)$. Show that $f$ is measurable if and only if $Y\in \cF$, $f^{-1}(\{-\infty\})\in\cF$, $f^{-1}(\{\infty\})\in \cF$, and $f$ is measurable when restricted to $Y$.  
\end{problem}

\begin{problem}
    Let $(\Omega,\cF)$ be a measurable space and $f_i$, $i\in \bbN$ be measurable functions from $(\Omega,\cF)$ to $(\bbR,\cB_\bbR)$. Show that the set 
    \begin{equation*}
        \{\omega\in \Omega: \lim_{n\to \infty} f_n(\omega) = 0\}
    \end{equation*}
    is measurable, that is, it belongs to $\cF$.
    (Hint: recall $\lim_{n\to \infty} f_n(\omega) = 0$ means that for all $m>0$ there exists $N>0$ such that $|f_n(\omega)|\leq 1/m$ for all $n\geq N$. Can you now write the set as countable union and intersections of measurable sets.)
\end{problem}

\begin{problem} Consider a function $f:\bbR\to \bbR$. Show that $f$ is Lebesgue measurable if and only if there exists a Borel measurable function $g$ such that $f\equiv g$ almost everywhere.
\end{problem}

\begin{problem} Show that if $f:\bbR\to\bbR$ is monotone, then $f$ is Borel-measurable.
\end{problem}

\begin{problem} Suppose that $f:\bbR\to \bbR$ is an homeomorphism, that is $f$ is continuous with continuous inverse. Show that $f$ maps Borel measurable sets to Borel measurable sets.   
\end{problem}

\begin{problem} Let $(\Omega,\cF,\mu)$ be a complete measure space. Show that 
\begin{itemize}
    \item if $f$ is measurable and $f\equiv g$ $\mu$-a.e. then $g$ is measurable.
    \item if $f_n$, $n\in \bbN$ are measurable and $f_n\to f$ $\mu$-a.e.\ then $f$ is measurable.
\end{itemize}
\end{problem}

\begin{problem}
    Let $(\Omega,\cF)$ be a measurable space and $f:\Omega\to\overline{\bbR}$ be a function. 
    Define $f^+(\omega) = \max(f(\omega),0)$ and  $f^-(\omega) = -\min(f(\omega),0)$.
    Show that $f$ is measurable if and only if $f^+$ and $f^{-}$ are measurable.
\end{problem}

\begin{problem}   Let $(\cX,\cF)$ and $(\cY,\cG)$ be two measurable spaces. Let $E\in \cF\otimes \cG$, and for $y\in \cY$ define the $y$-section of $E$ to be 
    \begin{equation*}
        E_y = \{x\in \cX:\, (x,y)\in E\}, 
    \end{equation*}
    Similarly define for $x\in \cX$ the $x$-section of $E$ to be 
    \begin{equation*}
        E_x = \{y\in \cY:\, (x,y)\in E\}.
    \end{equation*}
    Show that $E_y$ is measurable for all $y\in \cY$.
\end{problem}

\begin{problem} Let $(\Omega,\cF)$ and $(E,\cG)$ be measurable spaces, and let $f :\Omega \to E$
 be a function. Suppose that $A\in \cF$, we say that $f$ is measurable on $A$ if 
 $f^{-1}(B)\cap A\in \cF$ for all $B\in  \cG$. Show that $f$ is measurable on $A\in \cF$ if and only if $f|_A$ is $(\cF_A,\cG)$-measurable, where $\cF_A = \{A\cap B: B\in \cF\}$. 
 Show that if $f$ is $(\cF,\cG)$-measurable, then $f$ is measurable on $A$ for all $A\in \cF$.
\end{problem}

\begin{problem} Let $(\Omega,\cF)$ be a measurable space, $f :\Omega \to \overline{\bbR}$ and $\cY=f^{-1}(\bbR)$. Then $f$ is measurable if and only if $f^{-1} ( \{ -\infty\}) \in \cF$, $f^{-1} ( \{ \infty\}) \in  \cF$, and $f$ is measurable when restricted on $\cY$. 
\end{problem}

\begin{problem}
    If a function $Y_A : A \to E$ is $(\cF_A, \cG)$-measurable, and $p \in E$, then
the extension $Y$ defined by
\begin{equation*}
    Y (\omega) := \begin{cases}
Y_A(\omega),  &\omega \in A,\\
p, &\omega\notin A,
    \end{cases}
\end{equation*}
is $(\cF, \cG)$-measurable.
\end{problem}

\begin{problem}
    Does there exist a non-measurable function $f \geq 0$ such that
    $\sqrt{f}$ is measurable?
\end{problem}

\begin{problem}
    Show that if $f:\bbR\to \bbR$ is continuous almost everywhere, then $f$ is Lebesgue-measurable.
\end{problem}

\begin{problem}
    Is the following true of false?
    If $f : \bbR \to \bbR$ is differentiable, then $f'$ is 
\end{problem}

\begin{problem}(Convergence in measure/probability) Let $(\Omega,\cF,\bbP)$ be a probability space and $X_n:\Omega \to \bbR$, $n\in \bbN$ be random variables. We say that $X_n$ converges in probability to a random variable $X$, if for all $\epsilon>0$
    \begin{equation*}
        \bbP(|X_n-X|>\epsilon) \to 0,\qquad\text{as $n\to \infty$}.
    \end{equation*}  
\begin{itemize}
    \item Show that if $X_n$ converges to a random variable $X$ almost surely, then $X_n$ converges in probability to $X$.
    \item(*) Show that if $X_n$ converges to a random variable $X$ in probability, then there exists a subsequence of $(X_n)$ which converges almost surely to $X$. (\emph{Hint: use Borell-Cantelli lemma.}) 
\end{itemize}
\end{problem}