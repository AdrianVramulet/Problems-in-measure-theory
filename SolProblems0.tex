\begin{solution}[1.1]{Kempen, S.F.M.} 
    
\noindent a) To be proven: $A\cap (B\cup C) = (A\cap B) \cup (A\cap C)$.

\noindent $``\subseteq"$ Let $x \in A\cap(B\cup C)$ then $x\in A$ and $x\in B\cup C$, which means $x\in A$ and ($x\in B$ or $x\in C$). 
\begin{itemize}
	\item If $x\in A$ and $x\in B$, then $x\in A\cap B$ so also $x\in (A\cap B) \cup (A\cap C)$.
	\item If $x\in A$ and $x\in C$, then $x\in A\cap C$ so also $x\in (A\cap B) \cup (A\cap C)$.
\end{itemize}

\noindent $``\supseteq"$ Let $x\in (A\cap B) \cup (A\cap C)$ then $x\in A\cap B$ or $x\in A\cap C$.
\begin{itemize}
	\item If $x\in A\cap B$, then $x\in A$ and $x\in B$ so $x\in B\cup C$ so also $x\in A\cap (B \cup C)$.
	\item If $x\in A\cap C$, then $x\in A$ and $x\in C$ so $x\in B\cup C$ so also $x\in A\cap (B \cup C)$.
\end{itemize}

\noindent b) To be proven: $A\cup(B\cap C) = (A\cup B) \cap (A\cup C)$.

\noindent $``\subseteq"$ Let $x\in A\cup (B\cap C)$ then $x\in A$ or $x\in B\cap C$, which means $x\in A$ or ($x\in B$ and $x\in C$). 
\begin{itemize}
	\item If $x\in A$, then $x\in A\cup B$ and $x\in A\cup C$ so also $x\in (A\cup B) \cap (A\cup C)$.
	\item If $x\in B$ and $x\in C$, then $x\in A\cup B$ and $x\in A\cup C$ so also $x\in (A\cup B) \cap (A\cup C)$.
\end{itemize}
\noindent $``\supseteq"$ Let $x\in (A\cup B) \cap (A\cup C)$ then $x\in A\cup B$ and $x\in A\cup C$, which means $(x\in A$ or $x\in B$) and ($x\in A$ or $x\in C$). 
\begin{itemize}
	\item If $x\in A$ then definitely $x\in A\cup (B\cap C)$.
	\item If $x\notin A$ then $x\in B$ and $x\in C$ which means $x\in B\cap C$ so also $x\in A\cup (B\cap C)$.
\end{itemize}
\end{solution}


\begin{solution}[1.2]{Kempen, S.F.M.} To be proven: $A\cap (A\cup B) = A$. 

\noindent $``\subseteq"$ Let $x\in A\cap (A\cup B)$, then $x\in A$ so we are done.

\noindent $``\supseteq"$ Let $x\in A$, then also ($x\in A$ or $x\in B$) is true, therefore $x\in A\cup B$. So $x\in A\cap (A\cup B)$. 
\end{solution}

\begin{solution}[1.5]{Kempen, S.F.M.} To be proven: $\Omega \setminus \left(\bigcup_{i\in I} A_i \right) = \bigcap_{i\in I} \Omega \setminus A_i$. 

\noindent $``\subseteq"$ Let $x\in \Omega \setminus \left(\bigcup_{i\in I} A_i \right)$ then $x\in \Omega$ and $x\notin \bigcup_{i\in I} A_i$, so for all $i \in I $ holds $x\notin A_i$. Then for all $i \in I$ we have $x\in \Omega\setminus A_i$. Since this is true for any $i\in I$, we can write $x\in \bigcap_{i\in I} \Omega \setminus A_i$. 

\noindent $``\supseteq"$ Let $x\in \bigcap_{i\in I} \Omega \setminus A_i$ then for all $i\in I$ we have $x\in \Omega\setminus A_i$, so $x\in \Omega$ and $x\notin A_i$. Since this holds for all $i\in I$, we can write $x\notin \bigcup_{i\in I} A_i $ and therefore $x\in \Omega \setminus \left(\bigcup_{i\in I} A_i \right)$.
\end{solution}

\begin{solution}[1.8]{Kempen, S.F.M.} 
    
\noindent a) The statement $f^{-1}(f(A)) = A$ is not true since $f$ is not assumed to be injective. As a counterexample, take $\Omega = \{0,1\}, E = \{0\}, f(\{0\}) = f(\{1\}) = \{0\}, A = \{0\}$ then $f(A) = \{0\}$ and $f^{-1}(f(A)) = f^{-1}(\{0\}) = \{0,1\} \neq A$.

\noindent b) The statement $f(f^{-1}(H)) = H$ is not true since f is not assumed to be surjective. As a counterexample, take $\Omega = \{1\}$, $E = \{1,2\}$, $f(\{1\}) = \{1\}$, $H = \{1,2\}$ then $f^{-1}(H) = f^{-1}(\{1,2\}) = \{1\}$ and $f(f^{-1}(H)) = f(\{1\}) = \{1\} \neq H$.
\end{solution}

\begin{solution}[1.10]{Beurskens, T.P.J.}
Let $n \in \bbN$, and define $K = \{1, \ldots, n\}$.
By definition of the supremum, we then have
$$\sum_{i = 1}^n a_i = \sum_{i \in K} a_i \leq \sup \left( \sum_{i \in K} a_i : K \subseteq I, K~\text{finite} \right) = \sum_{i \in I} a_i.$$
Letting $n \to \infty$, we get
\[
\lim_{n \to \infty} \sum^n_{i = 1} a_i \leq \sum_{i \in I} a_i.
\]
Next, let $K \subseteq \bbN$ be finite, so that $K \subset \{1, \ldots, n\}$ for some $n \in \bbN$.
Note that $n \geq \sup K$.
We get
\[
\sum_{i \in K} a_i \leq \sum_{i \in \{1,\ldots, n\}} a_i = \sum_{i = 1}^n  a_i \leq \lim_{n\to\infty} \sum_{i = 1}^n  a_i.
\]
Since this holds for arbitrary finite $K$, it holds for all finite $K$.
Thus we get
\[
\sum_{i \in I} a_i = \sup \left( \sum_{i \in K} a_i : K \subseteq I, K~\text{finite} \right) \leq \lim_{n\to\infty} \sum_{i = 1}^n  a_i.
\]
Using both inequalities, we see that indeed $\sum_{i \in I} a_i = \lim_{n \to \infty} \sum^n_{i = 1} a_i$. 

We are left with showing that the sum $\sum_{i = 1}^\infty a_i$ does not depend on the ordering of the elements in the sequence $(a_i)$. This follows immediately from the fact that for any index set $I$
\begin{equation*}
	\sum_{i \in I} a_i = \sup\Big\{\sum_{i\in K} a_i:\, K\subseteq I,\text{ $K$ finite}\Big\}
\end{equation*}
is completely blind to any ordering of $I$, in fact $I$ is possibly not even ordered. To be more precise, if $\sigma:I\to I$ is a bijection, then
\begin{equation*}
	\begin{aligned}
	\sum_{i \in I} a_{\sigma(i)} & =  \sup\Big\{\sum_{i\in K} a_{\sigma(i)}:\, K\subseteq I,\text{ $K$ finite}\Big\} \\ 
	& = \sup\Big\{\sum_{i\in \sigma^{-1}(K)} a_i:\, \sigma^{-1}(K)\subseteq I,\text{ $\sigma^{-1}(K)$ finite}\Big\} = \sum_{i \in I} a_{i},
	\end{aligned}
\end{equation*}
where we used that $K\subseteq I$ is finite if and only if $\sigma^{-1}(K)$ is finite.
So, from this observation and the first part of the problem, one has that for any bijection $\sigma:\bbN\to \bbN$
\begin{equation*}
	\sum^\infty_{i = 1} a_i = \sum_{i\in \bbN} a_i = \sum_{i\in \bbN} a_{\sigma(i)} = \sum_{i = 1}^\infty a_{\sigma(i)},
\end{equation*}
where the summation in the middle is with respect the new notion.
\end{solution}

\begin{solution}[1.11]{Vrămuleț, A. Ș.}\\
Note for every $a_i > 0$, there exists $n \in \bbN$ such that
\[a_i > \frac{1}{n}\] (choose $n := \max(\ceil{a_i} + 1, \ceil*{\frac{1}{a_i}} + 1)).$\\
We can now define  	
\[
A := \{i \in I | a_i > 0 \} = \bigcup_{n \geq 1} J_n.
\]
Suppose for a contradiction $J_n$ is infinite. Since 
\[
\sum_{i \in I} a_i = \sum_{i \in I, a_i > 0} a_i,
\]
it suffices to show the supremum of (finite) partial sums on RHS is infinity (to derive a contradiction). \\
Let $n \in \bbN$. Since $J_n$ is infinite, then for every $N_0 \in \bbN$ we may choose a finite subset $K \subset I$ with at least $N_0$ elements, such that
\[
\sum_{i \in K} a_i > \frac{N_0}{n}. 
\]
Since the set of partial sums is unbounded above, then it does not admit a finite supremum. Hence the supremum is $\infty$ (over the extended real line).\\
This contradicts that the arbitrary indexed sum is finite. So each $J_n$ is finite.\\
Finally, since countable union of finite sets is countable, we conclude $A$ is countable.

\end{solution}

\begin{solution}[1.12]{Bakker, A.}
	Proof by contradiction. Suppose $\sum_{i\in I}a_i < \infty$, then by Problem 1.11 we have that the set $I$ contains at most a countable number of elements $i$ with $a_i$ positive. This, together with the fact that $a_i>0$ for all $i\in I$, contradicts that there are uncountable many elements in $I$. Hence  $\sum_{i\in I}a_i = \infty$.
\end{solution}

\begin{solution}[1.13]{Vrămuleț, A. Ș}\\
	Note by the previous exercise that since the indexed sum over $\Omega$ is finite (here, 1), then there are countably many positive (and nonzero) $p_{\omega}$.\\
	Suffices to show $\bbP(\emptyset) = 0$ and that $\bbP$ is countably additive. 
	Then 
	\[
	\bbP(\emptyset) = \sum_{\omega \in \emptyset} p_w = 0.
	\]
	Now let $(A_i)$ be a sequence of pairwise disjoint subsets of $\Omega$. We show 
	\[
	\bbP( \bigcup_{i = 1}^{\infty} A_i) = \sum_{i = 1}^\infty \bbP(A_i).
	 \]
   
   Let $A := \bigcup_{i = 1}^{\infty} A_i$. Since the sequence of sets is pairwise disjoint, then
   \begin{align*}
   \bbP(A) &= \sum_{\omega \in A} p_w \\
   			&= \sup\{\sum_{\omega \in A_0} p_w | \text{ finite} A_0 \subset A \} \\
   			&= \sup\{\sum_{i = 1}^n \sum_{\omega \in A_i} p_w | \text{ finite} A_i \subset A_0 \subset A  \} \quad \text{(after relabelling)}   \\
   			&= \sum_{i = 1}^{\infty} \sum_{\omega \in A_i} p_w \quad \text{(definition of series as limit of the (increasing) sequence of partial sums)} \\
   			&= \sum_{i = 1}^{\infty} \bbP(A_i).
   \end{align*}
   Note the suprema above are finite, since the indexed sum over $\Omega$ is 1 and we are summing subsets of omega. Hence $\bbP$ is a measure on $(\Omega, 2^{\Omega})$.
\end{solution}

\begin{solution}[1.14]{Vrămuleț, A. Ș}\\
Let $x \in A$. Then $x$ is an interior point of A. So there exists $r > 0$ such that 
\[
B(x,r) \subset A, \text{ where } B(x,r) = (x-r,x+r).
\]
Since $\bbQ$ is dense in $\bbR$, then there exist rationals $p$ and $q$ such that
\[
p \in (x-r,x) \text{ and } q \in (x,x+p).
\]
So $(p,q) \subset (x-r,x+r) \subset A$ and $x \in (p,q)$. Note there are countably many such intervals (since $\bbQ$ is countable, so is $\bbQ \times \bbQ$). So A is the union of countably many intervals with rational endpoints of the form $(p,q)$.
\end{solution}

\begin{solution}[1.15]{Vrămuleț, A. Ș}\\
Note from before that an indexed sum of positive reals is finite if and only if there are countably many nonzero reals (in the sum). So in the case that at least one of $I$ and $J$ is uncountable, we actually have countable sums. \\

\noindent So the sums can be split as uncountable sum of zeros plus countable sum of positive reals. So in all cases the sums reduce to summing over countable sets.
It's known that the result holds (by considering (absolute) convergence of the series)). Note the result also holds when the series converges to $\infty$ (over the extended real line).
\end{solution}