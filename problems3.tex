\begin{problem}
    Let $(\Omega,\cF,\mu)$ be a measure space. Suppose that $A,N\in \cF$ and $\mu(N) =0$. Show that $\mu(A\cup N) = \mu(A)$.
\end{problem}

\begin{problem} Suppose that $(\Omega,\cF,\mu)$ is a measure space and let $N$ be a null set. Show that for all $M\subseteq N$, $M$ is a null set.
\end{problem}
    
 \begin{problem}
    Let $(N_n)_{n\in \bbN}$ be null sets in a measure space $(\Omega,\cF,\mu)$. Show that $\cup_{n\in \bbN} N_n$ is a null set.
\end{problem}

\begin{problem}
    Let $(f_n)_{n\in \bbN}$ be a sequence of function. Show that $f_n = 0$ almost everywhere for all $n\in \bbN$ if and only if almost everywhere $f_n=0$ for all $n\in \bbN$. (Careful with the quantifiers!)
\end{problem}

\begin{problem} Let $(\Omega,\cF,\bbP)$ be a probability, we say that $A\in\cF$ happens almost surely if $\Omega\setminus A$ is a null set for $\bbP$.
    \begin{itemize}
        \item Show that $A\in \cF$ happens almost surely if and only if $\bbP(A) = 1$.
        \item Assume now that $(A_n)_{n\in \bbN}\subseteq \cF$ is such that $A_n$ happens almost surely for all $n\in \bbN$. Show that $\bbP(\cap_{n\in \bbN} A_n) = 1$.
    \end{itemize} 
\end{problem}

\begin{problem}
        Let $E\subseteq \bbR$ be a Lebesgue measurable set and let $V$ be the Vitali set. Show that if $E \subseteq V$, then $E$ is a null-set.
\end{problem}

\begin{problem}
    Let $(\Omega,\cF,\mu)$ be a measure space and $(\Omega,\overline{\cF},\overline{\mu})$ its completion. Show that $\overline{A}\in \overline{\cF}$ if and only if there is $A\in \cF$ such that $A\Delta \overline{A}$ is a null set.
\end{problem}

\begin{problem}(*) Let $E\subseteq \bbR$ be a Lebesgue measurable set such that $\cL(E)>0$. Show that there exists $N\subseteq E$ not Lebesgue measurable. (Hint: assume first $E\subseteq (0,1)$ and look at $V\cap E$ where $V$ is the Vitali set.) 
    \end{problem}
    
\begin{problem}(The Cantor set) The Lebesgue null sets include not only the countable sets but also many sets having the cardinality of the continuum. The Cantor set $C$ is the set of all $x \in [0, 1]$ that have a base-$3$ expansion
        \begin{equation*}
            x = \sum_{j=1}^\infty \frac{a_j}{3^j},\quad \text{with $a_j\in\{0,2\}$ for all $j\in \bbN$.}
        \end{equation*} 
        Thus $C$ is obtained from $[0,1]$ by removing the open middle third $(1/3,2/3)$, then removing the middle thirds $(1/9,2/9)$ and $(7/9,8/9)$ of the remaining intervals and so forth.
        Show that 
        \begin{itemize}
            \item $C$ is compact and with zero Lebesgue measure.
            \item $\mathrm{Card}(C) = \mathrm{Card}([0,1])$. Hint: consider the so called Cantor function, for $x\in C$, $x=\sum_{j=1}^\infty \frac{a_j}{3^j}$, define
            \begin{equation*}
                f(x) =\sum_{j=1}^\infty \frac{b_j}{2^j},\quad b_j=a_j/2.
            \end{equation*}
            \item (*) Show that $C$ has empty interior and is totally disconnected (that is for all $x<y\in C$ there is $z\in(x,y)$ such that $z\notin C$). Moreover $C$ has no isolated points.
\end{itemize}
\end{problem}
        

\begin{problem}
    Show that for any Lebesgue measurable set $E\subseteq \bbR$ and any real number $\lambda\in \bbR$, $\cL(E+\lambda) = \cL(E)$ and $\cL(\lambda E) = |\lambda| \cL(E)$.
\end{problem}

\begin{problem}
    Let $(\Omega,\cF,\bbP)$ be a probability space. We say that two $\sigma$-algebras $\cA_1, \cA_2\subseteq \cF$ are independent if
    \begin{equation*}
        \bbP(A_1\cap A_2) = \bbP(A_1)\bbP(A_2),\quad \forall A_1\in\cA_1,\,\forall A_2\in \cA_2.
    \end{equation*}
    Suppose that $\cE_1$, $\cE_2$ are $\pi$-systems generating $\cA_1$ and $\cA_2$ respectively. Show that $\cA_1$ and $\cA_2$ are independent if and only if
    \begin{equation*}
        \bbP(A_1\cap A_2) = \bbP(A_1)\bbP(A_2),\quad \forall A_1\in\cE_1,\,\forall A_2\in \cE_2.
    \end{equation*} 
\end{problem}

\begin{problem}
    Are the following true of false?
    \begin{itemize}
        \item If $A$ is an open subset of $[0, 1]$, then $\cL^1(A) = \cL^1(\overline{A})$, where $\overline{A}$ is the closure of the set.
        \item If $A$ is a subset of $[0, 1]$ such that $\cL^1(\mathrm{int}(A)) = \cL^1(\overline{A})$, then $A$ is measurable. Here $\mathrm{int}(A)$ denotes the interior of the set $A$.
    \end{itemize}
\end{problem}

\begin{problem}
Show that if $A \subset [0, 1]$ and $\cL^1(A) > 0$, then there are $x$ and $y$
in $A$ such that $|x - y|$ is an irrational number.
\end{problem}

\begin{problem}
    Let $(\Omega,\cF,\mu)$ be a measure space and $(\Omega,\overline{\cF},\overline{\mu})$ its completion. Show that a set $A\subseteq \Omega$ is a $\mu$-null set if and only if $A$ is a $\overline{\mu}$-null set.
\end{problem}