\begin{solution}[3.11]{Beerens, L.}
    \begin{itemize}
        \item Let $x_1,x_2\in\mathbb{R}$ with $x_1<x_2$. Then
        $$
            F(x_2) = \mu((-\infty,x_1]\cup (x_1,x_2]) = \mu((-\infty, x_1)) + \mu((x_1,x_2])\geq\mu((-\infty, x_1)) = F(x_1).
        $$
        Therefore, $F$ is non-decreasing.
        
        \item Consider $\lim_{x\downarrow x_0}F(x)$. This can be written as 
        $$
            \lim_{x\downarrow x_0}\mu((-\infty,x]).
        $$
        We know that this limit exists, so we can look at a sequence that converges to $x_0$ from above instead. Let $(x_n)$ be a strictly decreasing sequence in $\mathbb{R}$ such that $x_n\rightarrow x_0$. For all $n\in\mathbb{N}$, let $A_n:=(-\infty,x_n]$. Then
        $$
            \lim_{x\downarrow x_0}\mu((-\infty,x]) = \lim_{n\rightarrow \infty}\mu((-\infty,x_n]) = \lim_{n\rightarrow \infty}\mu(A_n).
        $$
        Since $A_n$ is a decreasing sequence of measurable sets such that $\mu(A_1)$ is finite, we find that
        $$
            \lim_{n\rightarrow \infty}\mu(A_n) = \mu\left( \bigcap_{i=1}^\infty A_i \right) = \mu((-\infty, x_0]),
        $$
        which completes the proof.
        
        \item We shall turn towards $\lim_{x\rightarrow -\infty}F(x)$. We know that this limit exists, so we can look at a sequence that diverges to $-\infty$ instead. Let the sequence $(x_n)$ be defined by $x_n = -n$. For all $n\in\mathbb{N}$, let $A_n:=(-\infty,x_n]$. Then
        $$
            \lim_{x\downarrow x_0}\mu((-\infty,x]) = \lim_{n\rightarrow \infty}\mu((-\infty,x_n]) = \lim_{n\rightarrow \infty}\mu(A_n).
        $$
        Since $A_n$ is a decreasing sequence of measurable sets such that $\mu(A_1)$ is finite, we find that
        $$
            \lim_{n\rightarrow \infty}\mu(A_n) = \mu\left( \bigcap_{i=1}^\infty A_i \right) = \mu(\varnothing) = 0,
        $$
        which completes the proof.
        
        \item Finally, we consider $\lim_{x\rightarrow \infty}F(x)$. We know that this limit exists, so we can look at a sequence that diverges to $\infty$ instead. Let $(x_n)$ be a sequence in $\mathbb{R}$ defined by $x_n = n$. For all $n\in\mathbb{N}$, let $A_n:=(-\infty,x_n]$. Then
        $$
            \lim_{x\rightarrow \infty}F(x) = \lim_{n\rightarrow \infty}F(x_n).
        $$
        Since $A_n$ is an increasing sequence of measurable sets, we find that
        $$
            \lim_{n\rightarrow \infty}\mu(A_n) = \mu\left( \bigcup_{i=1}^\infty A_i \right) = \mu(\mathbb{R}),
        $$
        which completes the proof.
    \end{itemize}
\end{solution}

\begin{solution}[3.14]{Beerens, L.}
    Let $(\Omega,\mathcal{F})$ be a measure space. Let $\mu:\mathcal{F}\rightarrow[0,\infty]$ be a set function which is finitely additive and such that $\mu(\varnothing) = 0$. If $\mu$ is a measure, then it is continuous from below, since that is a property of measures. Conversely, suppose that $\mu$ is continuous from below. Note that $\mu$ assigns to each set $A\in\mathcal{F}$ a nonnegative extended real number $\mu(A)$. It was already assumed that $\mu(\varnothing) =0$, so we will proceed by proving that $\mu$ is $\sigma$-additive. Let $A_1, A_2,\hdots$ be a sequence of mutually disjoint elements of $\mathcal{F}$. For all $n\in\mathcal{N}$, let 
    $$
        B_n = \bigcup_{i=1}^n A_i.
    $$
    Since $\mathcal{F}$ is a $\sigma$-algebra, we find that all $B_n$ are sets in $\mathcal{F}$.
    From finite additivity, we find that for all $n\in\mathbb{N}$ we have
    $$
        \mu(B_n) = \mu\left( \bigcup_{i=1}^n A_i \right) = \sum_{i=1}^n \mu(A_i). 
    $$
    Note that $(B_n)$ is an increasing sequence of measurable sets in $\mathcal{F}$. By continuity from below it follows that
    $$
        \mu\left( \bigcup_{i=1}^\infty B_i \right) = \lim_{n\rightarrow\infty}\mu(B_n) = \sum_{i=1}^\infty\mu(A_i).
    $$
    Therefore,
    $$
        \mu\left( \bigcup_{i=1}^\infty A_i \right) = \mu\left( \bigcup_{i=1}^\infty B_i \right) = \sum_{i=1}^\infty\mu(A_i).
    $$
    Thus $\mu$ is $\sigma$-additive and we can conclude that $\mu$ is a measure.
\end{solution}

\begin{solution}[3.21]{Beerens, L.}
    Let $F:\mathbb{R}\rightarrow\mathbb{R}$ be a non-decreasing, right-continuous function. Let $\nu_F$ be the unique measure on $(\mathbb{R},\mathcal{B}_\mathbb{R})$ such that $\nu_F((a,b])=F(b)-F(a)$ for all $a<b$. For all $i\in\mathbb{Z}$, let
    $$
        A_i:= (i,i+1]
    $$
    and let $\mathcal{A}$ be the collection of all these sets $A_i$. Note that $\mathcal{A}$ is a countable cover of $\mathbb{R}$. By the assumption we know that for all $i\in\mathbb{Z}$ we have
    $$
        \nu(A_i)  = F(i+1) - F(i).
    $$
    Since $F$ takes values in $\mathbb{R}$, we find that $\nu(A_i)$ is finite for all $i\in\mathbb{Z}$. Therefore, $\mathcal{A}$ is a countable cover of $\mathbb{R}$ that consists of sets with finite measure. Thus we conclude that $\mu$ is $\sigma$-finite.
\end{solution}