\documentclass{easyclass}

\usepackage{lecture_notes}

%%%%%%%%%%%%%%%%%%%%%%%%%%%%%%%%%%%%%%%%%%%%%%%%%%%%%%%%%%%%%%%%%%%%%%%%%%%%%%%%%%%%%%%%%%%%%%%%%%%
%										For leaving comments									  %
%%%%%%%%%%%%%%%%%%%%%%%%%%%%%%%%%%%%%%%%%%%%%%%%%%%%%%%%%%%%%%%%%%%%%%%%%%%%%%%%%%%%%%%%%%%%%%%%%%%

%Name:			XXX
%Description:	Adds a piece of text in blue, surrounded by [] brackets. 
%				#1: the person the comment is addressed to
%				#2: the person the comment is from
%				#3: the comment
% Usage:			Write \XXX{collaborator}{myName}{myComment} to write [myComment] addressed to [collaborator]
\newcommand{\XXX}[3]{{\color{blue} \textbf{ [#1:  #3 \textit{ -#2-} ]}}}

%%%%%%%%%%%%%%%%%%%%%%%%%%%%%%%%%%%%%%%%%%%%%%%%%%%%%%%%%%%%%%%%%%%%%%%%%%%%%%%%%%%%%%%%%%%%%%%%%%%
%									     BibTeX commands 									      %
%%%%%%%%%%%%%%%%%%%%%%%%%%%%%%%%%%%%%%%%%%%%%%%%%%%%%%%%%%%%%%%%%%%%%%%%%%%%%%%%%%%%%%%%%%%%%%%%%%%

%Name:			Swap
%Description:	Command for properly typesetting "van der" and related expressions in Dutch and German names in the
%				bibliography. It swaps [van der] and [Lastname] in [van der Lastname] so that ordering will be performed on 
%				[Lastname] instead of [van der].
%Usage:			Write \swap{Lastname}{~van~der~}, Firstname instead of [van der Lastname, Firstname] in the author header
%				of the bibtex entry.

\newcommand*{\swap}[2]{\hspace{-0.5ex}#2#1}


%%%%%%%%%%%%%%%%%%%%%%%%%%%%%%%%%%%%%%%%%%%%%%%%%%%%%%%%%%%%%%%%%%%%%%%%%%%%%%%%%%%%%%%%%%%%%%%%%%%
%											Document											  %
%%%%%%%%%%%%%%%%%%%%%%%%%%%%%%%%%%%%%%%%%%%%%%%%%%%%%%%%%%%%%%%%%%%%%%%%%%%%%%%%%%%%%%%%%%%%%%%%%%%


\begin{document}
\begin{titlepage}
    \university{TU Eindhoven}
    \courseid{2WAG0}
    \title{Problems in Measure theory}
    \author{Alberto Chiarini}
    \version{Version 0.1, \today}
    \maketitle
\end{titlepage}

\begin{chapquote}[30pt]{Kurt Friedrichs}
``What I don't like about measure theory is that you have to say ``almost everywhere'' almost everywhere''
\end{chapquote}
\vfill
\textbf{Disclaimer:}

\vspace{0.5cm} 
This is a broad selection of exercises for the course \emph{Measure, integration and probability theory}. They are meant to accompany the lecture notes and give you the opportunity to exercise. If you wish to have your solution checked, send it in \LaTeX, and we will correct and polish it together, so that it can be featured in this notes in the ``Solutions'' part.

These collection of exercises are still in progress and they might contain small typos. If you see any or if you think that the statement of the problems is not yet crystal clear, feel free to drop a line. The most efficient way is to send an email to me, \href{mailto:a.chiarini@tue.nl}{a.chiarini@tue.nl}. All comments and suggestions will be greatly appreciated.

\newpage

\tableofcontents

% Part I

\part{Problems}

\chapter{Warming up}
In this section we will review some of the basic set operations which will be much needed in the sequel. 

\begin{problem} Let $A$, $B$ and $C$ be sets, show that 
    \begin{equation*}
        A\cap (B\cup C) = (A\cap B) \cup (A\cap C),\quad  A\cup (B\cap C) = (A\cup B) \cap (A\cup C).
    \end{equation*}
\end{problem}

\begin{problem} Let $A$, $B$ be sets, show that $A\cap (A\cup B) = A$.    
\end{problem}

\begin{problem} Let $A,B\subseteq \Omega$. We define the symmetric difference to be 
    \begin{equation*}
        A\Delta B = (A\setminus B)\cup (B\setminus A).
    \end{equation*}
    Show that $A\cup B$ is the disjoint union of $A\Delta B$ and $A\cap B$. 
\end{problem}

\begin{problem} Let $A,B\subseteq \Omega$, show that 
    \begin{equation*}
    \Omega\setminus (A\cup B) = (\Omega\setminus A)\cap(\Omega\setminus B).
    \end{equation*}
\end{problem}

\begin{problem} (De Morgan's law) Let $I$ be any index set and let $\{A_i\}_{i\in I}\subseteq 2^\Omega$ be a family subsets of $\Omega$. Show that 
    \begin{equation*}
        \Omega\setminus \bigg(\bigcup_{i\in I} A_i\bigg) = \bigcap_{i\in I} \Omega\setminus A_i. 
    \end{equation*}
\end{problem}


\begin{problem} Let $f:\Omega \to E$ be some function. Recall that for any $D\subseteq \Omega$ the \emph{image} of $D$ under $f$ is the set
    \begin{equation*}
        f(D) = \{f(x) :\,x\in D\},
    \end{equation*}
    Let $A,B\subseteq \Omega$. Show that
    \begin{itemize}
        \item $f(A\cap B) \subseteq f(A)\cap f(B)$,
        \item $f(A\cup B) = f(A)\cup f(B)$.
    \end{itemize}
    Find an example where $f(A\cap B) \neq f(A)\cap f(B)$. Is it true that $f(\Omega\setminus A) = E \setminus f(A)$?
\end{problem}

\begin{problem}Let $f:\Omega \to E$ be some function. Recall that for any $F\subseteq E$ the \emph{inverse image} of $F$ under $f$ is the set 
    \begin{equation*}
        f^{-1}(F) = \{x:\,f(x)\in F\}.
    \end{equation*}
    Let $H,K\subseteq E$. Show that, taking the inverse image commutes with the set operations:
\begin{itemize}
        \item $f^{-1}(H\cap K) = f^{-1}(H)\cap f^{-1}(K)$,
        \item $f^{-1}(H\cup K) = f^{-1}(H)\cup f^{-1}(K)$,
        \item $f^{-1}(E\setminus H) = \Omega \setminus f^{-1}(H)$.
\end{itemize}
\end{problem}

\begin{problem} Let $f:\Omega \to E$ be some function.
    \begin{itemize}
        \item Let $A\subseteq \Omega$. Is it true that $f^{-1}(f(A)) = A$? Provide a proof or a counterexample.
        \item Let $H\subseteq E$.  Is it true that $f(f^{-1}(E)) = E$? Provide a proof or a counterexample.
    \end{itemize}
\end{problem}

\begin{problem} Recall that given a set $\Omega$, $2^{\Omega}$ denotes the set of all subsets of $\Omega$. Suppose $\Omega = \{0,1\}$, list all the elements of $2^{\Omega}$. What is $|2^{\Omega}|$, where $|\cdot|$ denotes the number of elements of a set? Suppose that $|\Omega|<\infty$, what is $|2^{\Omega}|$ in this case? 
\end{problem}

\begin{problem} Let $\{a_i\}_{i\in I} \subseteq [0,\infty]$, where $I$ is an arbitrary (index) set. Recall that their sum is defined by
    \begin{equation*}
        \sum_{i\in I} a_i = \sup\Big\{\sum_{i\in K} a_i:\, K\subseteq I,\text{ $K$ finite}\Big\}.
    \end{equation*}
Now, suppose that $I = \bbN$. Show that the above definition agrees with the standard one, that is
\begin{equation*}
    \sum_{i\in I} a_i = \lim_{n\to\infty} \sum_{i=1}^n a_i.
\end{equation*}
Show that the value of the series does not depend on the ordering of the elements in the sequence. That is, if $\sigma:\bbN\to \bbN$ is a bijection, then 
\begin{equation*}
    \sum_{i=1}^\infty a_i = \sum_{i=1}^\infty a_{\sigma(i)}.
\end{equation*}
\end{problem}

\begin{problem} Let $\{a_i\}_{i\in I} \subseteq [0,\infty)$, where $I$ is an arbitrary (index) set. Suppose that
    \begin{equation*}
        \sum_{i\in I} a_i < \infty.
    \end{equation*}
Show that the set $J_n = \{i\in I:\, a_i>1/n\}$ is finite. Conclude that the set of $i\in I$ such that $a_i>0$ is at most countable. 
\end{problem}

\begin{problem} Let $\{a_i\}_{i\in I} \subseteq (0,\infty)$ be a family of \emph{positive} real numbers, where $I$ is an (index) set with uncountably many elements.
Show that
 \begin{equation*}
     \sum_{i\in I} a_i = \infty.
 \end{equation*}
\end{problem}

\begin{problem} (*) Let $\Omega$ be a non-empty set and $p_\omega\in [0,1]$, $\omega\in \Omega$ be real numbers such that 
\begin{equation*}
    \sum_{\omega\in \Omega} p_\omega = 1.
\end{equation*}    
Define the set function $\bbP:2^\Omega \to [0,1]$ by
\begin{equation*}
    \bbP(A) = \sum_{\omega\in A} p_\omega.
\end{equation*}    
Show that $\bbP$ is a measure on $2^\Omega$.
\end{problem}

\begin{problem} (*)
    Let $A\subset \bbR$ be an open set. Show that $A$ is the union of at most countable many intervals. (\emph{Hint:} define for all $x\in A$ the interval $I_x = \bigcup_{\text{$I$ interval:} x\in I \subseteq A} I$ to be the largest interval contained in $A$ containing $x$)
\end{problem}

\begin{problem}
    Let $I$ and $J$ be two index sets and $a_{i,j}$, $i\in I$ and $j\in J$ be non-negative real numbers. Show that 
    \begin{equation*}
        \sum_{i\in I} \sum_{j\in J} a_{i,j} = \sum_{j\in J} \sum_{i\in I} a_{i,j}. 
    \end{equation*}
\end{problem}



\chapter{Measurable sets and \texorpdfstring{$\sigma$}{}-algebras}


\begin{problem}
    Show that there is no $\sigma$-algebra with an odd number of elements.
\end{problem}

\begin{problem}
Let $(\Omega,\cF)$ be a measurable space and $A,B\in \cF$. Show, starting from the definition of $\sigma$-algebra, that $A\cup B$, $A\cap B$, $A\setminus B$, and $A\Delta B$ all belong to $\cF$.
\end{problem}

\begin{problem}
    Let $(\Omega,\cF)$ be a measurable space and $A_1, A_2,\ldots$ be a sequence of sets in $\cF$. Define the following sets
    \begin{equation*}
        \limsup_{n\to\infty} A_n = \bigcap_{n\in \bbN} \bigcup_{m\geq n} A_m,\qquad \liminf_{n\to\infty} A_n = \bigcup_{n\in \bbN} \bigcap_{m\geq n} A_m.
    \end{equation*}
    Show that:
    \begin{itemize}
        \item $(\limsup_{n\to\infty} A_n)^c = \liminf_{n\to\infty} A_n^c$,
        \item $\liminf_{n\to\infty} A_n \in \cF$ and $\limsup_{n\to\infty} A_n \in \cF$,
        \item $\liminf_{n\to\infty} A_n \subseteq \limsup_{n\to\infty} A_n$.
        \item  $\limsup_{n\to\infty} A_n = \{\omega \in \Omega:\, \omega\in A_n\text{ for infinitely many $n$}\}$.
        \item $\liminf_{n\to\infty} A_n = \{\omega \in \Omega:\, \text{$\exists m\in \bbN$ such that $\omega\in A_n$ for all $n\geq m$}\}$.
    \end{itemize}
\end{problem}

\begin{problem}
    Let $\Omega$, $E$ be non-empty, $\cG$ a $\sigma$-algebra on $E$ and $f:\Omega\to E$. Show that 
    \begin{equation*}
        \cF = \{f^{-1}(B):\,B\in \cG\},
    \end{equation*}
    is a $\sigma$-algebra on $\Omega$.
\end{problem}

\begin{problem}
    Let $(\Omega,\cF)$ be a measurable space and $(A_n)_{n\in \bbN}$ a collection of sets in $\cF$. Show that:
    \begin{itemize}
        \item There are $(E_n)_{n\in \bbN}\subseteq \cF$ mutually disjoint such that $\cup_{n\in \bbN} A_n = \cup_{n\in \bbN} E_n$.
        \item There are $(F_n)_{n\in \bbN}\subseteq \cF$ such that $F_n\subseteq F_{n+1}$ for all $n\in \bbN$ amd $\cup_{n\in \bbN} A_n = \cup_{n\in \bbN} E_n$.
    \end{itemize} 
\end{problem}

\begin{problem}
    Let $\Omega$ be a non-empty set and $\cF$ a non-empty collection of subsets of $\Omega$ which is closed under taking complements and finite unions (such a collection is called an \emph{algebra}). Show that $\cF$ is a $\sigma$-algebra if and only if it is closed under countable increasing unions (i.e., if $\{A_n\}\subseteq \cF$ and $A_1\subseteq A_2\subseteq \ldots$, then $\cup_{n\in \bbN} A_n \in \cF$).
\end{problem}
        

\begin{problem}(Restriction of $\sigma$-algebra)
    Let $\cF$ be a $\sigma$-algebra of subsets of $\Omega$. Suppose that $A\subseteq \Omega$ is non-empty. Show that
    \begin{equation*}
        \cF_A = \{B\cap A:\, B\in \cF\}
    \end{equation*}
    is a $\sigma$-algebra on $A$.
\end{problem}

\begin{problem}(Extension of $\sigma$-algebra) Let $(\Omega,\cF)$ be a measurable space, and let $K$ be some non-empty set such that $\Omega\cap K = \varnothing$. Define $\overline{\Omega} = \Omega \cup K$ and $\overline{\cF} = \sigma(\cF\cup K)$ be a $\sigma$-algebra on $\overline{\Omega}$. Show that $\overline{\cF} = \{A\subseteq \overline{\Omega}:\, A\cap \Omega\in \cF\}$.
    
\end{problem}

\begin{problem} Let $\Omega$ be a infinite non-empty set. 
    \begin{itemize}
        \item Define the collection of sets $\cF = \{A\subseteq \Omega:\, \text{$A$ is countable or $\Omega\setminus A$ is countable}\}$. Is $\cF$ a $\sigma$-algebra? Prove or disprove.
        \item Define the collection of sets $\cF = \{A\subseteq \Omega:\, \text{$A$ is finite or $\Omega\setminus A$ is finite}\}$. Is $\cF$ a $\sigma$-algebra? Prove or disprove.
    \end{itemize}
\end{problem}

\begin{problem}
    Let $\cF$ and $\cG$ be $\sigma$-algebras on $\Omega$. Show that $\cF\cap\cG$ is a $\sigma$-algebra. Prove or disprove whether $\cF\cup \cG$ is in general a $\sigma$-algebra.
\end{problem}


\begin{problem} Let $\cF_n$, $n\in \bbN$ be $\sigma$-algebras on $\Omega$ such that $\cF_{n}\subseteq \cF_{n+1}$ for all $n\in \bbN$ (such a sequence $\{\cF_n\}$ is called a \emph{filtration}). 
    \begin{itemize}
        \item Show that $\bigcup_{n\in \bbN} \cF_n$ is an algebra.
        \item Is $\bigcup_{n\in \bbN} \cF_n$ a $\sigma$-algebra? Consider $\Omega=\bbN$ and $\cF_n = \sigma(\{A:\,A\subseteq \bbN \cap \{1,\ldots n\}\})$).
    \end{itemize}
\end{problem}

\begin{problem}
    Let $\cE\subseteq \cA$ be two collections of sets. Show that $\sigma(\cE) \subseteq\sigma(\cA)$. 
\end{problem}

\begin{problem}(Product sigma algebra) Let $\Omega_1$ and $\Omega_2$ be two non-empty sets, and let $\cF_1$ and $\cF_2$ be $\sigma$-algebras on $\Omega_1$ and $\Omega_2$ respectively.
Consider the \emph{product $\sigma$-algebra} on $\Omega_1\times \Omega_2$
\begin{equation*}
    \cF_1\otimes\cF_2 := \sigma(\{A_1\times A_2:\, A_1\in \cF_1, A_2\in \cF_2\}.
\end{equation*}
Suppose that $\cF_1$ is generated by $\cA_1$ and $\cF_2$ is generated by $\cA_2$. Show that  $ \cF_1\otimes\cF_2$ is generated by $A_1\times A_2$ with $A_1\in \cA_1$ and $A_2\in \cA_2$.
\end{problem}

\begin{problem}
    Show that the Borel $\sigma$-algebra $\cB_{\bbR^2}$ equals $\cB_{\bbR}\otimes \cB_{\bbR}$.
\end{problem}

\begin{problem}
    Show that the Borel $\sigma$-algebra on $\bbR$ is generated by each of the following:
    \begin{enumerate}[i.] 
        \item the open intervals: $\cA_1 = \{(a,b):\, a<b\}$,
        \item the closed intervals: $\cA_2 = \{[a,b]:\, a < b\}$,
        \item the half open intervals $\cA_3 = \{[a,b):\, a < b\}$ or $\cA_4 = \{(a,b]:\, a < b\}$,
        \item the open rays: $\cA_5 = \{(a,\infty):\, a\in \bbR\}$ or $\cA_6 = \{(-\infty,a):\, a\in \bbR\}$,
        \item the closed rays: $\cA_7 = \{[a,\infty):\, a\in \bbR\}$ or $\cA_8 = \{(-\infty,a]:\, a\in \bbR\}$.
    \end{enumerate}
\end{problem}

\begin{problem}
    Recall that $\overline{\bbR} = \bbR\cup\{\pm\infty\}$ is the extended real line. Also recall that 
    \begin{equation*}
        \cB_{\overline{\cB}} = \{A\subseteq \overline{\cB}:\, A\cap \bbR \in \cB_{\bbR}\}
    \end{equation*}
    is a $\sigma$-algebra on $\overline{\bbR}$. Show that $\cB_{\overline{\cB}}$ is generated by the family of closed rays $\cA= \{[-\infty,a]:\,a\in \bbR\}$.
\end{problem}

\begin{problem}
    Let $\cF$ be an infinite $\sigma$-algebra. 
    \begin{itemize}
        \item Show that $\cF$ contains an infinite sequence of disjoint sets.
        \item (*) Show that $\mathrm{Card}(\cF)\geq \mathrm{Card}([0,1])$. \\(\emph{Hint:} think about binary representation of numbers in $[0,1]$).
    \end{itemize}
\end{problem}

\begin{problem}
    Show that $\Lambda$ is a $\lambda$-system on $\Omega$ if and only if
    \begin{enumerate}[I.]
        \item $\Omega\in \Lambda$,
        \item if $A,B\in \Lambda$ and $A\subseteq B$, then $B\setminus A\in \Lambda$,
        \item if $A_1,A_2,\ldots$ is a sequence of subsets in $\Lambda$ such that $A_n \subseteq A_{n+1}$ for all $n\in \bbN$, then
        \begin{equation*}
            \bigcup_{n\in \bbN} A_n \in \Lambda.
        \end{equation*}
    \end{enumerate}
\end{problem}


\begin{problem}
    Let $\Lambda$ be a $\lambda$-system. Show that $\varnothing\in \Lambda$.
\end{problem}

\begin{problem}
    Let $\cA$ be both a $\lambda$-system and a $\pi$-system. Show that $\cA$ is a $\sigma$-algebra.
\end{problem}

\chapter{Measures}

\begin{problem} Let $(\Omega,\cF,\mu)$ be a measure space. Show that for all $E,F\in \cF$ such that $E\subseteq F$, one has $\mu(E)\leq \mu(F)$. 
\end{problem}

\begin{problem} Let $(\Omega,\cF,\mu)$ be a measure space. Show that for all $E,F\in \cF$ then
\begin{equation*}
    \mu(E\cup F) = \mu(E) + \mu(F) - \mu(E\cap F).
\end{equation*}    
Conclude that $\mu(E\cup F) \leq \mu(E) + \mu(F)$.
\end{problem}

% \begin{problem}
% Let $(\Omega,\cF,\mu)$ be a measure space and $(A_n)_{n\in \bbN} \subseteq \cF$. Show that 
% \begin{equation*}
%     \mu\bigg(\bigcup_{n\in \bbN} A_n\bigg) \leq \sum_{n=1}^\infty \mu(A_n).
% \end{equation*}
% \end{problem}

\begin{problem}
    Let $(\Omega,\cF)$ be a measurable space and $\mu:\cF\to [0,\infty]$ be a set function. Assume that $\mu(\varnothing)=0$, $\mu$ is finitely additive and continuous from below. Show that $\mu$ is a measure.
\end{problem}

\begin{problem}
    Consider a non-empty uncountable set $\Omega$ and the $\sigma$-algebra 
    \begin{equation*}
        \cF=\{A\subseteq \Omega:\, \text{$A$ is countable or $\Omega\setminus A$ is countable}\}. 
    \end{equation*}
    We define the set function $\mu : \cF \to [0,\infty]$ by $\mu(E)=0$ if $E$ is countable, and $\mu(E)=1$ if $\Omega\setminus E$ is countable. Show that $\mu$ is a measure on $(\Omega,\cF)$.
\end{problem}

\begin{problem}
Let $\Omega$ be an infinite set and $\cF = 2^\Omega$. Define $\mu$ on $\cF$ by $\mu(E)=0$ if $E$ is finite and $\mu(E)=\infty$ if $E$ is not finite. Show that $\mu$ is finitely additive but not a measure.     
\end{problem}

\begin{problem}
    Let $(\Omega,\cF)$ be a measurable space and $\mu:\cF\to [0,1]$ be an additive set function, that is, $\mu(A\cup B) = \mu(A) + \mu(B)$ for all $A,B\in \cF$ such that $A\cap B = \varnothing$. Show that $\mu(\varnothing) = 0$.
\end{problem}

\begin{problem}(Inclusion-exclusion) Let $(\Omega,\cF,\mu)$ be a finite measure space. Let $A_1,A_2,A_3\in \cF$. Show that
\begin{multline*}
    \mu(A_1\cup A_2 \cup A_3) = \mu(A_1) + \mu(A_2) + \mu(A_3) \\- \mu(A_1\cap A_2) - \mu(A_2\cap A_3) - \mu(A_3\cap A_1) + \mu(A_1\cap A_2\cap A_3).   
\end{multline*}
Let $A_1,\ldots,A_n\in \cF$. Show that 
\begin{equation*}
    \mu\Big(\bigcup_{i=1}^n A_i \Big) = \sum_{j=1}^n (-1)^{j-1} \sum_{\substack{I\subseteq \{1,\ldots,n\}\\ |I| = j}} \mu\Big(\bigcap_{i\in I} A_i\Big).
\end{equation*}
\end{problem}

\begin{problem}
    Let $(\Omega,\cF,\mu)$ be a finite measure space.
    \begin{itemize}
        \item If $E,F\in \cF$ and $\mu(E\Delta F) =0$, then $\mu(E) = \mu(F)$.
        \item Define $\rho(E,F) = \mu(E\Delta F)$ for all $E,F\in \cF$. Show that $\rho(E,F) \leq \rho(E,G)+\rho(G,F)$ for all $E,F,G\in \cF$.
    \end{itemize}
\end{problem}

\begin{problem} If $\mu_1,\ldots ,\mu_n$ are measures on a measurable space $(\Omega,\cF)$, and $a_1,\ldots, a_n$ are non-negative real numbers, then $\mu:=\sum_{i=1}^n a_i \mu_i$ is a measure on $\cF$. Moreover $\mu$ is $\sigma$-finite if $\mu_i$ is $\sigma$-finite for all $i=1,\ldots,n$. 
\end{problem}



\begin{problem}
    Let $(\Omega,\cF,\mu)$ be a measure space and $\cG\subseteq \cF$ be a $\sigma$-algebra. Show that $(\Omega,\cG,\mu)$ is a measure space.
\end{problem}

\begin{problem}
    Let $\mu$ be a finite measure on $(\bbR,\cB_{\bbR})$. Define $F:\bbR\to\bbR$ by 
    \begin{equation*}
        F(x) = \mu((-\infty,x]), \quad \forall x\in \bbR.
    \end{equation*}
    Show that $F$ is non-decreasing, right-continuous, and 
    \begin{equation*}
        \lim_{x\to-\infty} F(x) = 0,\quad \lim_{x\to\infty} F(x) = \mu(\bbR).
    \end{equation*}
\end{problem} 

\begin{problem}
Find a measure space $(\Omega,\cF,\mu)$ and a decreasing sequence $B_1\supseteq B_2\supseteq\ldots \in \cF$ such that $\lim_{n\to\infty}\mu(B_n) > \mu(\cap_{n\in \bbN} B_n)$.    
\end{problem}

\begin{problem}
    Let $(\Omega,\cF,\mu)$ be a measure space and $A\in \cF$. Show that the set function $\mu_A:\cF\to[0,\infty]$ defined by $\mu_A(B):=\mu(B\cap A)$ is a measure on $(\Omega,\cF)$.  
\end{problem}

\begin{problem}
    Let $(\Omega,\cF)$ be a measure space $\mu:\cF\to[0,\infty]$ be a set function which is finitely additive and such that $\mu(\varnothing)=0$. Show that $\mu$ is a measure on $(\Omega,\cF)$ if and only if $\mu$ is continuous from below. 
\end{problem}

\begin{problem}
    Let $(\Omega,\cF)$ be a measure space $\mu:\cF\to[0,\infty]$ be a set function which is finitely additive and such that $\mu(\Omega)<\infty$. Show that $\mu$ is a measure on $(\Omega,\cF)$ if and only if $\mu$ is continuous from above. 
\end{problem}


\begin{problem}
    Let $(\Omega,\cF,\mu)$ be a finite measure space. Let $(A_n)_{n\in \bbN}$ be a sequence in $\cF$. Recall that 
    \begin{equation*}
        \limsup_{n\to\infty} A_n = \bigcap_{n\in \bbN} \bigcup_{m\geq n} A_m,\qquad \liminf_{n\to\infty} A_n = \bigcup_{n\in \bbN} \bigcap_{m\geq n} A_m.
    \end{equation*} 
    Show that
    \begin{equation*}
        \limsup_{n\to\infty} \mu(A_n) \leq  \mu \Big(\limsup_{n\to\infty} A_n\Big),\qquad \liminf_{n\to\infty} \mu(A_n) \geq  \mu \Big(\liminf_{n\to\infty} A_n\Big).
    \end{equation*}
\end{problem}

\begin{problem}
    Let $(\Omega,\cF,\mu)$ be a $\sigma$-finite measure space. Let $\cE\subseteq \cF$ be a $\pi$-system such that there exists $E_1,E_2,\ldots \in \cE$ such that $E_n\uparrow \Omega$ and $\mu(E_n)<\infty$ for all $n\in \bbN$. Show that $\mu$ is uniquely determined by its values on $\cE$.
\end{problem}

\begin{problem}  Let $F:\bbR\to \bbR$ be a non-decreasing, right-continuous function. Let $\nu_F$ be the unique measure on $(\bbR,\cB_{\bbR})$ associated to $F$. Show that $\nu_F(\{x\}) = F(x)-F(x-)$ where we define 
    \begin{equation*}
        F(x-) := \lim_{y\uparrow x} F(y).
    \end{equation*} 
Conclude that it if $F$ is continuous, then $\nu_F(\bbQ) = 0$.
\end{problem}

\begin{problem} Let $\lambda$ be the unique measure on $(\bbR,\cB_{\bbR})$ such that $\lambda((a,b]) = b-a$ for all $a<b$. Show that $\lambda$ is translation invariant, that is $\lambda(A+x) = \lambda(A)$ for all $A\in \cB_{\bbR}$ and all $x\in \bbR$, where we write $A+x := \{a+x:\, a\in A\}$ for the translation of $A$ by $x$.
(\emph{Hint:} a solution can be obtained with the $\pi-\lambda$ theorem.)    
\end{problem}

\begin{problem} Let $\lambda$ be the unique measure on $(\bbR,\cB_{\bbR})$ such that $\lambda((a,b]) = b-a$ for all $a<b$. Show that $\lambda(\tau A) = |\tau|\lambda(A)$ for all $A\in \cB_{\bbR}$ and all $\tau\neq 0$, where we write $\tau A := \{\tau a:\, a\in A\}$ for the dilation of $A$ by $\tau$.
\end{problem}

\begin{problem}
    Let $F:\bbR\to \bbR$ be a non-decreasing, right-continuous function. Let $\nu_F$ be the unique measure on $(\bbR,\cB_{\bbR})$ such that $\nu_F((a,b]) = F(b)-F(a)$ for all $a<b$. Show that $\nu_F$ is $\sigma$-finite.
\end{problem}

\begin{problem}
    Let $(\Omega,\cF,\mu)$ be a measure space. Suppose that $A,N\in \cF$ and $\mu(N) =0$. Show that $\mu(A\cup N) = \mu(A)$.
\end{problem}

\begin{problem}
    Let $(\Omega,\bbF,\bbP)$ be a probability space. Let $(A_i)_{i\in \bbN}\subseteq \cF$ be a sequence such that $\bbP(A_i)=1$ for all $i\in\bbN$. Show that $\bbP(\cap_{i\in \bbN} A_i) = 1$.
\end{problem}

\begin{problem}
    Let $\mu$ be a finite measure on $(\bbR,\cB_{\bbR})$. Show that the set $\{x\in \bbR:\, \mu(\{x\})>0\}$ is at most countable.
\end{problem}

\begin{problem}(The Cantor set) The Lebesgue null sets include not only the countable sets but also many sets having the cardinality of the continuum. The Cantor set $C$ is the set of all $x \in [0, 1]$ that have a base-$3$ expansion
\begin{equation*}
    x = \sum_{j=1}^\infty \frac{a_j}{3^j},\quad \text{with $a_j\in\{0,2\}$ for all $j\in \bbN$.}
\end{equation*} 
Thus $C$ is obtained from $[0,1]$ by removing the open middle third $(1/3,2/3)$, then removing the middle thirds $(1/9,2/9)$ and $(7/9,8/9)$ of the remaining intervals and so forth.
Show that 
\begin{itemize}
    \item $C$ is compact and with zero Lebesgue measure.
    \item $\mathrm{Card}(C) = \mathrm{Card}([0,1])$. Hint: consider the so called Cantor function, for $x\in C$, $x=\sum_{j=1}^\infty \frac{a_j}{3^j}$, define
    \begin{equation*}
        f(x) =\sum_{j=1}^\infty \frac{b_j}{2^j},\quad b_j=a_j/2.
    \end{equation*}
    \item (*) Show that $C$ has empty interior and is totally disconnected (that is for all $x<y\in C$ there is $z\in(x,y)$ such that $z\notin C$). Moreover $C$ has no isolated points.
\end{itemize}
\end{problem}

\begin{problem}
    Let $E\subseteq \bbR$ be a Lebesgue measurable set and let $V$ be the Vitali set. Show that $E$ is a null-set.
\end{problem}

\begin{problem}
    Let $(A_n)_{n\in \bbN}$ be null sets in a measure space $(\Omega,\cF,\mu)$. Show that $\cup_{n\in \bbN} A_n$ is a null set.
\end{problem}

\begin{problem}
    Let $(\Omega,\cF,\mu)$ be a measure space. Show that a property holds for almost all $\omega\in \Omega$ if and only if there exists $B \in \cF$ such that $\mu(B)=0$ and the property holds for all $\omega$ in $\Omega\setminus B$.
\end{problem}

\begin{problem}
    Let $(\Omega,\cF,\bbP)$ be a probability space. Show that a property holds for almost all $\omega\in \Omega$ if and only if there exists $\Omega'\in \cF$ such that $\bbP(\Omega')=1$ and the property holds for all $\omega$ in $\Omega'$.
\end{problem}

\begin{problem}(*) Let $E\subseteq \bbR$ be a Lebesgue measurable set such that $\cL(E)>0$. Show that there exists $N\subseteq E$ not Lebesgue measurable. (Hint: assume first $E\subseteq (0,1)$ and look at $V\cap E$ where $V$ is the Vitali set.) 
\end{problem}



\chapter{Null sets, completion and independence}

\begin{problem}
    Let $(\Omega,\cF,\mu)$ be a measure space. Suppose that $A,N\in \cF$ and $\mu(N) =0$. Show that $\mu(A\cup N) = \mu(A)$.
\end{problem}

\begin{problem} Suppose that $(\Omega,\cF,\mu)$ is a measure space and let $N$ be a null set. Show that for all $M\subseteq N$, $M$ is a null set.
\end{problem}
    
 \begin{problem}
    Let $(N_n)_{n\in \bbN}$ be null sets in a measure space $(\Omega,\cF,\mu)$. Show that $\cup_{n\in \bbN} N_n$ is a null set.
\end{problem}

\begin{problem}
    Let $(f_n)_{n\in \bbN}$ be a sequence of function. Show that $f_n = 0$ almost everywhere for all $n\in \bbN$ if and only if almost everywhere $f_n=0$ for all $n\in \bbN$. (Careful with the quantifiers!)
\end{problem}

\begin{problem} Let $(\Omega,\cF,\bbP)$ be a probability, we say that $A\in\cF$ happens almost surely if $\Omega\setminus A$ is a null set for $\bbP$.
    \begin{itemize}
        \item Show that $A\in \cF$ happens almost surely if and only if $\bbP(A) = 1$.
        \item Assume now that $(A_n)_{n\in \bbN}\subseteq \cF$ is such that $A_n$ happens almost surely for all $n\in \bbN$. Show that $\bbP(\cap_{n\in \bbN} A_n) = 1$.
    \end{itemize} 
\end{problem}

\begin{problem}
        Let $E\subseteq \bbR$ be a Lebesgue measurable set and let $V$ be the Vitali set. Show that if $E \subseteq V$, then $E$ is a null-set.
\end{problem}

\begin{problem}
    Let $(\Omega,\cF,\mu)$ be a measure space and $(\Omega,\overline{\cF},\overline{\mu})$ its completion. Show that $\overline{A}\in \overline{\cF}$ if and only if there is $A\in \cF$ such that $A\Delta \overline{A}$ is a null set.
\end{problem}

\begin{problem}(*) Let $E\subseteq \bbR$ be a Lebesgue measurable set such that $\cL(E)>0$. Show that there exists $N\subseteq E$ not Lebesgue measurable. (Hint: assume first $E\subseteq (0,1)$ and look at $V\cap E$ where $V$ is the Vitali set.) 
    \end{problem}
    
\begin{problem}(The Cantor set) The Lebesgue null sets include not only the countable sets but also many sets having the cardinality of the continuum. The Cantor set $C$ is the set of all $x \in [0, 1]$ that have a base-$3$ expansion
        \begin{equation*}
            x = \sum_{j=1}^\infty \frac{a_j}{3^j},\quad \text{with $a_j\in\{0,2\}$ for all $j\in \bbN$.}
        \end{equation*} 
        Thus $C$ is obtained from $[0,1]$ by removing the open middle third $(1/3,2/3)$, then removing the middle thirds $(1/9,2/9)$ and $(7/9,8/9)$ of the remaining intervals and so forth.
        Show that 
        \begin{itemize}
            \item $C$ is compact and with zero Lebesgue measure.
            \item $\mathrm{Card}(C) = \mathrm{Card}([0,1])$. Hint: consider the so called Cantor function, for $x\in C$, $x=\sum_{j=1}^\infty \frac{a_j}{3^j}$, define
            \begin{equation*}
                f(x) =\sum_{j=1}^\infty \frac{b_j}{2^j},\quad b_j=a_j/2.
            \end{equation*}
            \item (*) Show that $C$ has empty interior and is totally disconnected (that is for all $x<y\in C$ there is $z\in(x,y)$ such that $z\notin C$). Moreover $C$ has no isolated points.
\end{itemize}
\end{problem}
        

\begin{problem}
    Show that for any Lebesgue measurable set $E\subseteq \bbR$ and any real number $\lambda\in \bbR$, $\cL(E+\lambda) = \cL(E)$ and $\cL(\lambda E) = |\lambda| \cL(E)$.
\end{problem}

\begin{problem}
    Let $(\Omega,\cF,\bbP)$ be a probability space. We say that two $\sigma$-algebras $\cA_1, \cA_2\subseteq \cF$ are independent if
    \begin{equation*}
        \bbP(A_1\cap A_2) = \bbP(A_1)\bbP(A_2),\quad \forall A_1\in\cA_1,\,\forall A_2\in \cA_2.
    \end{equation*}
    Suppose that $\cE_1$, $\cE_2$ are $\pi$-systems generating $\cA_1$ and $\cA_2$ respectively. Show that $\cA_1$ and $\cA_2$ are independent if and only if
    \begin{equation*}
        \bbP(A_1\cap A_2) = \bbP(A_1)\bbP(A_2),\quad \forall A_1\in\cE_1,\,\forall A_2\in \cE_2.
    \end{equation*} 
\end{problem}

\begin{problem}
    Are the following true of false?
    \begin{itemize}
        \item If $A$ is an open subset of $[0, 1]$, then $\cL^1(A) = \cL^1(\overline{A})$, where $\overline{A}$ is the closure of the set.
        \item If $A$ is a subset of $[0, 1]$ such that $\cL^1(\mathrm{int}(A)) = \cL^1(\overline{A})$, then $A$ is measurable. Here $\mathrm{int}(A)$ denotes the interior of the set $A$.
    \end{itemize}
\end{problem}

\begin{problem}
Show that if $A \subset [0, 1]$ and $\cL^1(A) > 0$, then there are $x$ and $y$
in $A$ such that $|x - y|$ is an irrational number.
\end{problem}

\begin{problem}
    Let $(\Omega,\cF,\mu)$ be a measure space and $(\Omega,\overline{\cF},\overline{\mu})$ its completion. Show that a set $A\subseteq \Omega$ is a $\mu$-null set if and only if $A$ is a $\overline{\mu}$-null set.
\end{problem}

\chapter{Measurable functions}

\begin{problem}
    Let $(\Omega,\cF)$ be a measure space. Let $A\subseteq \Omega$ and consider the indicator function $\indE{A} : \Omega\to \bbR$, defined by 
    \begin{equation*}
        \indE{A}(\omega) = 1, \text{ if $\omega\in A$},\quad  \indE{A}(\omega) = 0, \text{ if $\omega\notin A$}.
    \end{equation*}
    Show that $\indE{A}$ is $(\cF,\cB)$-measurable if and only if $A\in \cF$.
\end{problem}

\begin{problem}
    Let $(\Omega,\cF)$ be a measure space and $A,B\in \cF$. Show that
    \begin{equation*}
        \indE{A\cup B} = \indE{A} + \indE{B} - \indE{A\cap B},\qquad \indE{A\cap B} = \indE{A} \indE{B}. 
    \end{equation*}
\end{problem}

\begin{problem}
    Let $(\Omega,\cF,\bbP)$ be a probability space, and $X,Y:\Omega\to \bbR$ be random variables. Recall that $X$ and $Y$ are called independent if
    \begin{equation*}
        \bbP(A\cap B) = \bbP(A)\bbP(B),\qquad\forall A\in \sigma(X),\, \forall B\in \sigma(Y).
    \end{equation*} 
    Show that $X$ and $Y$ are independent if and only if 
    \begin{equation*}
        \bbP(X\geq s, Y\geq t) = \bbP(X\geq s) \bbP(Y\geq t),\qquad\forall s,t\in \bbR.
    \end{equation*} 
    (Hint: you might need the $\pi-\lambda$ Theorem.)
\end{problem}

\begin{problem}
    Let $(\Omega,\cF)$ be a measurable space and $f,g:\Omega\to \bbR$ be  $\cF$-measurable functions. Show that $\max(f,g)$, $\min(f,g)$ and $|f|$ are measurable. 
\end{problem}

\begin{problem} Let $(\Omega,\cF)$ be a measurable space and $E_1,\ldots,E_n\in \cF$ be measurable sets. Show that if $a_1,\ldots,a_n$ are real numbers, then
\begin{equation*}
    f = \sum_{i=1}^n a_i \indE{E_i}
\end{equation*}
is $\cF$-measurable.
\end{problem}

\begin{problem}
    Let $(\Omega,\cF)$ be a measurable space, $f:\Omega\to \overline{\bbR}$, and $Y = f^{-1}(\bbR)$. Show that $f$ is measurable if and only if $Y\in \cF$, $f^{-1}(\{-\infty\})\in\cF$, $f^{-1}(\{\infty\})\in \cF$, and $f$ is measurable when restricted to $Y$.  
\end{problem}

\begin{problem}
    Let $(\Omega,\cF)$ be a measurable space and $f_i$, $i\in \bbN$ be measurable functions from $(\Omega,\cF)$ to $(\bbR,\cB_\bbR)$. Show that the set 
    \begin{equation*}
        \{\omega\in \Omega: \lim_{n\to \infty} f_n(\omega) = 0\}
    \end{equation*}
    is measurable, that is, it belongs to $\cF$.
    (Hint: recall $\lim_{n\to \infty} f_n(\omega) = 0$ means that for all $m>0$ there exists $N>0$ such that $|f_n(\omega)|\leq 1/m$ for all $n\geq N$. Can you now write the set as countable union and intersections of measurable sets.)
\end{problem}

\begin{problem} Consider a function $f:\bbR\to \bbR$. Show that $f$ is Lebesgue measurable if and only if there exists a Borel measurable function $g$ such that $f\equiv g$ almost everywhere.
\end{problem}

\begin{problem} Show that if $f:\bbR\to\bbR$ is monotone, then $f$ is Borel-measurable.
\end{problem}

\begin{problem} Suppose that $f:\bbR\to \bbR$ is an homeomorphism, that is $f$ is continuous with continuous inverse. Show that $f$ maps Borel measurable sets to Borel measurable sets.   
\end{problem}

\begin{problem} Let $(\Omega,\cF,\mu)$ be a complete measure space. Show that 
\begin{itemize}
    \item if $f$ is measurable and $f\equiv g$ $\mu$-a.e. then $g$ is measurable.
    \item if $f_n$, $n\in \bbN$ are measurable and $f_n\to f$ $\mu$-a.e.\ then $f$ is measurable.
\end{itemize}
\end{problem}

\begin{problem}
    Let $(\Omega,\cF)$ be a measurable space and $f:\Omega\to\overline{\bbR}$ be a function. 
    Define $f^+(\omega) = \max(f(\omega),0)$ and  $f^-(\omega) = -\min(f(\omega),0)$.
    Show that $f$ is measurable if and only if $f^+$ and $f^{-}$ are measurable.
\end{problem}

\begin{problem}   Let $(\cX,\cF)$ and $(\cY,\cG)$ be two measurable spaces. Let $E\in \cF\otimes \cG$, and for $y\in \cY$ define the $y$-section of $E$ to be 
    \begin{equation*}
        E_y = \{x\in \cX:\, (x,y)\in E\}, 
    \end{equation*}
    Similarly define for $x\in \cX$ the $x$-section of $E$ to be 
    \begin{equation*}
        E_x = \{y\in \cY:\, (x,y)\in E\}.
    \end{equation*}
    Show that $E_y$ is measurable for all $y\in \cY$.
\end{problem}

\begin{problem} Let $(\Omega,\cF)$ and $(E,\cG)$ be measurable spaces, and let $f :\Omega \to E$
 be a function. Suppose that $A\in \cF$, we say that $f$ is measurable on $A$ if 
 $f^{-1}(B)\cap A\in \cF$ for all $B\in  \cG$. Show that $f$ is measurable on $A\in \cF$ if and only if $f|_A$ is $(\cF_A,\cG)$-measurable, where $\cF_A = \{A\cap B: B\in \cF\}$. 
 Show that if $f$ is $(\cF,\cG)$-measurable, then $f$ is measurable on $A$ for all $A\in \cF$.
\end{problem}

\begin{problem} Let $(\Omega,\cF)$ be a measurable space, $f :\Omega \to \overline{\bbR}$ and $\cY=f^{-1}(\bbR)$. Then $f$ is measurable if and only if $f^{-1} ( \{ -\infty\}) \in \cF$, $f^{-1} ( \{ \infty\}) \in  \cF$, and $f$ is measurable when restricted on $\cY$. 
\end{problem}

\begin{problem}
    If a function $Y_A : A \to E$ is $(\cF_A, \cG)$-measurable, and $p \in E$, then
the extension $Y$ defined by
\begin{equation*}
    Y (\omega) := \begin{cases}
Y_A(\omega),  &\omega \in A,\\
p, &\omega\notin A,
    \end{cases}
\end{equation*}
is $(\cF, \cG)$-measurable.
\end{problem}

\begin{problem}
    Does there exist a non-measurable function $f \geq 0$ such that
    $\sqrt{f}$ is measurable?
\end{problem}

\begin{problem}
    Show that if $f:\bbR\to \bbR$ is continuous almost everywhere, then $f$ is Lebesgue-measurable.
\end{problem}

\begin{problem}
    Is the following true of false?
    If $f : \bbR \to \bbR$ is differentiable, then $f'$ is 
\end{problem}

\begin{problem}(Convergence in measure/probability) Let $(\Omega,\cF,\bbP)$ be a probability space and $X_n:\Omega \to \bbR$, $n\in \bbN$ be random variables. We say that $X_n$ converges in probability to a random variable $X$, if for all $\epsilon>0$
    \begin{equation*}
        \bbP(|X_n-X|>\epsilon) \to 0,\qquad\text{as $n\to \infty$}.
    \end{equation*}  
Show that if $X_n$ converges to a random variable $X$ in probability, then there exists a subsequence of $(X_n)_{n\in \bbN}$ which converges almost surely to $X$. (\emph{Hint: use Borell-Cantelli Lemma.)}
\end{problem}

\chapter{Integration}

\begin{problem} Consider the measure space $(\cX, 2^{\cX},\delta_x)$ where $\delta_x$ is the delta measure at $x\in \cX$. Let $f:\cX\to \overline{\bbR}$ be a non-negative measurable function. Show that 
\begin{equation*}
    \int_{\cX} f \,\De \delta_x = f(x).
\end{equation*}  
\end{problem}

\begin{problem} Let $(\Omega,\cF,\mu)$ be a measure space, $A\in \cF$, and $f:\Omega\to \overline{\bbR}$ be a non-negative  function. We say that $f$ is measurable on $A$ if $f|_{A}$ is $\cF_A$-measurable (recall that $\cF_A = \{A\cap B:\,B\in \cF\}$). Show that $f \indE{A}$ is $\cF$ measurable and also 
\begin{equation*}
    \int_\Omega f  \indE{A} \,\De \mu = \int_A f \,\De \mu_A, 
\end{equation*}
where $\mu_A$ is the restriction of $\mu$ to $\cF_A$.
\end{problem}

\begin{problem}
    Let $(\Omega,\cF,\mu)$ be a measure space and $f:\Omega\to \overline{\bbR}$ be a non-negative measurable function
    \begin{itemize}
        \item Let $A,B\in \cF$ be such that $A\subseteq B$. Show that
        \begin{equation*}
            \int_{A} f\,\De \mu \leq  \int_{B} f\,\De \mu.
        \end{equation*}
        \item Let $A,B\in \cF$ be such that $A\cap B = \varnothing$. Show that 
        \begin{equation*}
            \int_{A\cup B} f\,\De \mu =  \int_{A} f\,\De \mu + \int_{B} f\,\De \mu.
        \end{equation*}
    \end{itemize}
\end{problem}

\begin{problem}
     Suppose that $(f_n)\subseteq L^+$, $f_n\to f$ pointwise and $\lim_{n\to\infty} \int f_n\, \De\mu =  \int f\, \De\mu <\infty$. Show, without the dominated convergence theorem, that 
     for all $A\in \cF$
     \begin{equation*}
         \lim_{n\to\infty} \int_A f_n\,\De \mu = \int_A f\,\De \mu.
     \end{equation*}  
     Show that the conclusion might fail if $ \int f\, \De\mu = \infty$.
\end{problem}

\begin{problem}
    Let $(\Omega,\cF,\mu)$ be a measure space and  $(\Omega,\overline{\cF},\overline{\mu})$ its completion. Suppose that $f$ is integrable with respect to $\mu$. Show that $f$ is integrable with respect to $\overline{\mu}$ and in particular 
    \begin{equation*}
        \int_{\Omega} f\,\De \mu =  \int_{\Omega} f\,\De \overline{\mu}.
    \end{equation*}
    (\emph{Hint: start with simple functions.})
\end{problem}

\begin{problem} Let $(\Omega,\cF,\mu)$ be a measure space and suppose that $(f_{n})$ is a sequence of measurable functions such that $f_1\geq f_2\geq\ldots \geq 0$. Show using the MCT that if $f :=\lim_{n\to\infty} f_n$ and $\int_{\Omega} f_1<\infty$, then
\begin{equation*}
    \lim_{n\to\infty} \int_\Omega f_n\,\De \mu =  \int_\Omega f\,\De \mu.
\end{equation*}
Show that if $\int_{\Omega} f_1\,\De \mu = \infty$ the above conclusion might not hold.    
\end{problem}

\begin{problem}
    Let $(\Omega,\cF,\mu)$ be a measure space and $f\in L^+$. Show that if $\int f\,\De \mu <\infty$, then
    \begin{equation*}
        \lim_{n\to\infty} \int_\Omega f \ind{f\geq n} \,\De \mu = 0. 
    \end{equation*}
\end{problem}

\begin{problem} Let $(\Omega,\cF,\mu)$ be a measure space and assume that $f:\Omega\to \overline{\bbR}$ is integrable with respect to $\mu$. Show that $|f|<\infty$ $\mu$-almost everywhere.
    
\end{problem}

\begin{problem} Suppose that $f$ is a Lebesgue-integrable function on $\bbR$. Show that for all $z\in \bbR$
    \begin{equation*}
        \int_\bbR f(x)\,\De \cL^1(x) = \int_\bbR f(x+z)\,\De \cL^1(x).
    \end{equation*}
\end{problem}

\begin{problem}
    Assume Fatou's lemma and deduce the Monotone Convergence Theorem.
\end{problem}

\begin{problem} Let $(\Omega,\cF,\mu)$ be a measure space and assume that $f$ is integrable with respect to $\mu$ and $g$ is a measurable function such that $f = g$ $\mu$-almost everywhere. Show that $g$ is integrable with respect to $\mu$ and 
    \begin{equation*}
        \int_\Omega f\, \De \mu =  \int_\Omega g\, \De \mu.
    \end{equation*}
\end{problem}

\begin{problem}  Let $(\Omega,\cF,\mu)$ be a measure space and $f$ be integrable with respect to $\mu$. Show that the set $\{\omega\in \Omega :\, f(\omega) \neq 0 \}$ is a $\sigma$-finite set. 
\end{problem}

\begin{problem}
    Let $(\Omega,\cF,\mu)$ be a measure space and $f,g$ be integrable with respect to $\mu$. Then 
    \begin{equation*}
        \int_A f \,\De \mu = \int_A g \,\De \mu,\qquad \forall A\in\cF,
    \end{equation*}
    if and only if $f = g$ $\mu$-almost everywhere, and if and only if $\int_{\Omega}|f-g|\,\De \mu = 0$.
\end{problem}

\begin{problem}
    Let $(\Omega_1,\cF_1)$ and $(\Omega_2,\cF_2)$ be two measure spaces. Let $f:\Omega_1\times\Omega_2\to \overline{\bbR}$, $f\geq 0$ be a $\cF_1\otimes \cF_2$-measurable function. Show that for all $y\in \Omega_2$ the map $f_y : \Omega_1\to \overline{\bbR}$, defined by $f_y(x) = f(x,y)$ for all $x\in \Omega_1$, is $\cF_1$-measurable.

    \noindent (\emph{Hint:} use approximations with simple functions.)
\end{problem}

\begin{problem}
    Let $f$ be Lebesgue-integrable on the real line. Show that
    \begin{equation*}
        \lim_{n\to\infty} \frac{1}{2 n} \int_{[-n,n]} f \,\De\cL^1 = 0.
    \end{equation*}
    Show that the result need not be true if $f$ is not assumed to be integrable on $\bbR$.
\end{problem}

\begin{problem} Let $\mu$ be a measure on $(\bbR,\cB_\bbR)$ concentrated on $\bbZ$. This means that $\mu(B) = \mu(B\cap \bbZ)$ for all $B\in \cB_\bbR$, or equivalently 
$\mu = \sum_{n\in \bbZ}\mu_n \delta_n$, where $\mu_n$ are non-negative real numbers and $\delta_n$ is the Dirac measura at $n\in \bbZ$. Show that $\mu_n = \mu(\{n\})$ for all $n\in \bbZ$ and that  for all $f$ non-negative and measurable
\begin{equation*}
    \int_\bbR f\,\De \mu = \sum_{n\in \bbZ} \mu_n f(n).
\end{equation*}
\end{problem}

\begin{problem}Consider the functions
    \begin{equation*}
        f_1(x) = \begin{cases}
            +\infty, & \text{if } x=0,\\
            \log|x|, & \text{if } 0<|x|<1,\\
            0, & \text{if }|x|\geq 1,
        \end{cases}
        \qquad 
        f_2(x) = \begin{cases}
            \frac{1}{x^2-1}, & \text{if } |x|\neq 0,\\
            20, & \text{if } |x| = 0,\\
        \end{cases}
        \qquad f_3(x)\equiv 1.
    \end{equation*}
    Determine if these functions are integrable on $(\bbR,\cB_\bbR)$ in each of the following two cases, and if possible compute the value of the integral
    \begin{enumerate}[a)]
        \item $m =\cL^1$ is the Lebesgue measure,
        \item $m$ is defined by
        \begin{equation*}
            m(B) = \sum_{n\in \bbZ} \frac{1}{1+(n+1)^2} \delta_n(B)
        \end{equation*}
    \end{enumerate}
    for all $B\in \cB_\bbR$.
\end{problem}

\begin{problem} Let $f,g: [0,1]\to\bbR$ be two non-negative integrable functions. Show that $fg$ is not necessarily integrable.
\end{problem}

\begin{problem} Let $f$ be integrable on $(\bbR,\cB_\bbR)$ with respect to the Lebesgue measure. 
    \begin{enumerate}[a)]
        \item Show that for all $\epsilon>0$ there exists a simple function $g = \sum_{i=1}^k a_i \indE{A_i}$ with $A_1,\ldots, A_k$ bounded intervals, such that $\int_\bbR |f-g|\,\De \cL<\epsilon$.
        \item Show that all $\epsilon>0$ there exists a bounded continuous function $h$ such that $\int_\bbR |f-h|\,\De \cL<\epsilon$.
    \end{enumerate}
\end{problem}

\begin{problem} Determine the limits of 
    \begin{equation*}
        \int_0^n \Big(1-\frac{x}{n}\Big)^n e^{x/2}\,\De x,\quad\text{and}\quad \int_0^n \Big(1+\frac{x}{n}\Big)^n e^{-2x}\,\De x.
    \end{equation*}
\end{problem}

\begin{problem} Let $(\Omega,\cF,\mu)$ be a finite measure space. Let $(f_n)_n$ be a sequence of measurable functions such that $f_n\to f$ uniformly on $\Omega$. Show that if $f_n$ is integrable for all $n\in\bbN$ then $f$ is also integrable and
    \begin{equation*}
        \lim_{n\to\infty} \int_\Omega f_n \,\De \mu = \int_\Omega f \,\De \mu. 
    \end{equation*}
Recall that $f_n$ converges uniformly to $f$ on $\Omega$ if $\lim_{n\to\infty} \sup_{\omega\in \Omega} |f_n(\omega)-f(\omega)| = 0$.
Show that the result is in general false if $\mu(\Omega) = \infty$.
\end{problem}

\begin{problem}
    Let $f:\bbR\to \bbR$ be integrable with respect to the Lebesgue measure. Show that the map $x\mapsto \int_{[-\infty,x]} f\,\De \cL$ is continuous in $x\in \bbR$.
\end{problem}

% \begin{problem}
%     Let $f_n(x) = a e^{-n a x} - b e^{-b n x}$ with $0<a<b$. Show that
%     \begin{enumerate}
%         \item $\sum_{n=1}^\infty \int_0^\infty |f_n|\,\De \cL = \infty$,
%         \item $\sum_{n=1}^\infty \int_0^\infty f_n\,\De \cL = 0$,
%         \item $\sum_{n=1}^\infty \int_0^\infty f_n\,\De \cL$ is summable and $\int_0^\infty \sum_{n=1}^\infty f_n \,\De \cL = \log(b/a)$.
%     \end{enumerate}
% \end{problem}


% Part II

\part{Solutions}

\chapter{Solutions: Warming up}

\begin{solution}[1.1]{Kempen, S.F.M.} 
    
\noindent a) To be proven: $A\cap (B\cup C) = (A\cap B) \cup (A\cap C)$.

\noindent $``\subseteq"$ Let $x \in A\cap(B\cup C)$ then $x\in A$ and $x\in B\cup C$, which means $x\in A$ and ($x\in B$ or $x\in C$). 
\begin{itemize}
	\item If $x\in A$ and $x\in B$, then $x\in A\cap B$ so also $x\in (A\cap B) \cup (A\cap C)$.
	\item If $x\in A$ and $x\in C$, then $x\in A\cap C$ so also $x\in (A\cap B) \cup (A\cap C)$.
\end{itemize}

\noindent $``\supseteq"$ Let $x\in (A\cap B) \cup (A\cap C)$ then $x\in A\cap B$ or $x\in A\cap C$.
\begin{itemize}
	\item If $x\in A\cap B$, then $x\in A$ and $x\in B$ so $x\in B\cup C$ so also $x\in A\cap (B \cup C)$.
	\item If $x\in A\cap C$, then $x\in A$ and $x\in C$ so $x\in B\cup C$ so also $x\in A\cap (B \cup C)$.
\end{itemize}

\noindent b) To be proven: $A\cup(B\cap C) = (A\cup B) \cap (A\cup C)$.

\noindent $``\subseteq"$ Let $x\in A\cup (B\cap C)$ then $x\in A$ or $x\in B\cap C$, which means $x\in A$ or ($x\in B$ and $x\in C$). 
\begin{itemize}
	\item If $x\in A$, then $x\in A\cup B$ and $x\in A\cup C$ so also $x\in (A\cup B) \cap (A\cup C)$.
	\item If $x\in B$ and $x\in C$, then $x\in A\cup B$ and $x\in A\cup C$ so also $x\in (A\cup B) \cap (A\cup C)$.
\end{itemize}
\noindent $``\supseteq"$ Let $x\in (A\cup B) \cap (A\cup C)$ then $x\in A\cup B$ and $x\in A\cup C$, which means $(x\in A$ or $x\in B$) and ($x\in A$ or $x\in C$). 
\begin{itemize}
	\item If $x\in A$ then definitely $x\in A\cup (B\cap C)$.
	\item If $x\notin A$ then $x\in B$ and $x\in C$ which means $x\in B\cap C$ so also $x\in A\cup (B\cap C)$.
\end{itemize}
\end{solution}


\begin{solution}[1.2]{Kempen, S.F.M.} To be proven: $A\cap (A\cup B) = A$. 

\noindent $``\subseteq"$ Let $x\in A\cap (A\cup B)$, then $x\in A$ so we are done.

\noindent $``\supseteq"$ Let $x\in A$, then also ($x\in A$ or $x\in B$) is true, therefore $x\in A\cup B$. So $x\in A\cap (A\cup B)$. 
\end{solution}

\begin{solution}[1.5]{Kempen, S.F.M.} To be proven: $\Omega \setminus \left(\bigcup_{i\in I} A_i \right) = \bigcap_{i\in I} \Omega \setminus A_i$. 

\noindent $``\subseteq"$ Let $x\in \Omega \setminus \left(\bigcup_{i\in I} A_i \right)$ then $x\in \Omega$ and $x\notin \bigcup_{i\in I} A_i$, so for all $i \in I $ holds $x\notin A_i$. Then for all $i \in I$ we have $x\in \Omega\setminus A_i$. Since this is true for any $i\in I$, we can write $x\in \bigcap_{i\in I} \Omega \setminus A_i$. 

\noindent $``\supseteq"$ Let $x\in \bigcap_{i\in I} \Omega \setminus A_i$ then for all $i\in I$ we have $x\in \Omega\setminus A_i$, so $x\in \Omega$ and $x\notin A_i$. Since this holds for all $i\in I$, we can write $x\notin \bigcup_{i\in I} A_i $ and therefore $x\in \Omega \setminus \left(\bigcup_{i\in I} A_i \right)$.
\end{solution}

\begin{solution}[1.8]{Kempen, S.F.M.} 
    
\noindent a) The statement $f^{-1}(f(A)) = A$ is not true since $f$ is not assumed to be injective. As a counterexample, take $\Omega = \{0,1\}, E = \{0\}, f(\{0\}) = f(\{1\}) = \{0\}, A = \{0\}$ then $f(A) = \{0\}$ and $f^{-1}(f(A)) = f^{-1}(\{0\}) = \{0,1\} \neq A$.

\noindent b) The statement $f(f^{-1}(H)) = H$ is not true since f is not assumed to be surjective. As a counterexample, take $\Omega = \{1\}$, $E = \{1,2\}$, $f(\{1\}) = \{1\}$, $H = \{1,2\}$ then $f^{-1}(H) = f^{-1}(\{1,2\}) = \{1\}$ and $f(f^{-1}(H)) = f(\{1\}) = \{1\} \neq H$.
\end{solution}

\begin{solution}[1.10]{Beurskens, T.P.J.}
Let $n \in \bbN$, and define $K = \{1, \ldots, n\}$.
By definition of the supremum, we then have
$$\sum_{i = 1}^n a_i = \sum_{i \in K} a_i \leq \sup \left( \sum_{i \in K} a_i : K \subseteq I, K~\text{finite} \right) = \sum_{i \in I} a_i.$$
Letting $n \to \infty$, we get
\[
\lim_{n \to \infty} \sum^n_{i = 1} a_i \leq \sum_{i \in I} a_i.
\]
Next, let $K \subseteq \bbN$ be finite, so that $K \subset \{1, \ldots, n\}$ for some $n \in \bbN$.
Note that $n \geq \sup K$.
We get
\[
\sum_{i \in K} a_i \leq \sum_{i \in \{1,\ldots, n\}} a_i = \sum_{i = 1}^n  a_i \leq \lim_{n\to\infty} \sum_{i = 1}^n  a_i.
\]
Since this holds for arbitrary finite $K$, it holds for all finite $K$.
Thus we get
\[
\sum_{i \in I} a_i = \sup \left( \sum_{i \in K} a_i : K \subseteq I, K~\text{finite} \right) \leq \lim_{n\to\infty} \sum_{i = 1}^n  a_i.
\]
Using both inequalities, we see that indeed $\sum_{i \in I} a_i = \lim_{n \to \infty} \sum^n_{i = 1} a_i$. 

We are left with showing that the sum $\sum_{i = 1}^\infty a_i$ does not depend on the ordering of the elements in the sequence $(a_i)$. This follows immediately from the fact that for any index set $I$
\begin{equation*}
	\sum_{i \in I} a_i = \sup\Big\{\sum_{i\in K} a_i:\, K\subseteq I,\text{ $K$ finite}\Big\}
\end{equation*}
is completely blind to any ordering of $I$, in fact $I$ is possibly not even ordered. To be more precise, if $\sigma:I\to I$ is a bijection, then
\begin{equation*}
	\begin{aligned}
	\sum_{i \in I} a_{\sigma(i)} & =  \sup\Big\{\sum_{i\in K} a_{\sigma(i)}:\, K\subseteq I,\text{ $K$ finite}\Big\} \\ 
	& = \sup\Big\{\sum_{i\in \sigma^{-1}(K)} a_i:\, \sigma^{-1}(K)\subseteq I,\text{ $\sigma^{-1}(K)$ finite}\Big\} = \sum_{i \in I} a_{i},
	\end{aligned}
\end{equation*}
where we used that $K\subseteq I$ is finite if and only if $\sigma^{-1}(K)$ is finite.
So, from this observation and the first part of the problem, one has that for any bijection $\sigma:\bbN\to \bbN$
\begin{equation*}
	\sum^\infty_{i = 1} a_i = \sum_{i\in \bbN} a_i = \sum_{i\in \bbN} a_{\sigma(i)} = \sum_{i = 1}^\infty a_{\sigma(i)},
\end{equation*}
where the summation in the middle is with respect the new notion.
\end{solution}

\begin{solution}[1.11]{Vrămuleț, A. Ș.}\\
Note for every $a_i > 0$, there exists $n \in \bbN$ such that
\[a_i > \frac{1}{n}\] (choose $n := \max(\ceil{a_i} + 1, \ceil*{\frac{1}{a_i}} + 1)).$\\
We can now define  	
\[
A := \{i \in I | a_i > 0 \} = \bigcup_{n \geq 1} J_n.
\]
Suppose for a contradiction $J_n$ is infinite. Since 
\[
\sum_{i \in I} a_i = \sum_{i \in I, a_i > 0} a_i,
\]
it suffices to show the supremum of (finite) partial sums on RHS is infinity (to derive a contradiction). \\
Let $n \in \bbN$. Since $J_n$ is infinite, then for every $N_0 \in \bbN$ we may choose a finite subset $K \subset I$ with at least $N_0$ elements, such that
\[
\sum_{i \in K} a_i > \frac{N_0}{n}. 
\]
Since the set of partial sums is unbounded above, then it does not admit a finite supremum. Hence the supremum is $\infty$ (over the extended real line).\\
This contradicts that the arbitrary indexed sum is finite. So each $J_n$ is finite.\\
Finally, since countable union of finite sets is countable, we conclude $A$ is countable.

\end{solution}

\begin{solution}[1.12]{Bakker, A.}
	Proof by contradiction. Suppose $\sum_{i\in I}a_i < \infty$, then by Problem 1.11 we have that the set $I$ contains at most a countable number of elements $i$ with $a_i$ positive. This, together with the fact that $a_i>0$ for all $i\in I$, contradicts that there are uncountable many elements in $I$. Hence  $\sum_{i\in I}a_i = \infty$.
\end{solution}

\begin{solution}[1.13]{Vrămuleț, A. Ș}\\
	Note by the previous exercise that since the indexed sum over $\Omega$ is finite (here, 1), then there are countably many positive (and nonzero) $p_{\omega}$.\\
	Suffices to show $\bbP(\emptyset) = 0$ and that $\bbP$ is countably additive. 
	Then 
	\[
	\bbP(\emptyset) = \sum_{\omega \in \emptyset} p_w = 0.
	\]
	Now let $(A_i)$ be a sequence of pairwise disjoint subsets of $\Omega$. We show 
	\[
	\bbP( \bigcup_{i = 1}^{\infty} A_i) = \sum_{i = 1}^\infty \bbP(A_i).
	 \]
   
   Let $A := \bigcup_{i = 1}^{\infty} A_i$. Since the sequence of sets is pairwise disjoint, then
   \begin{align*}
   \bbP(A) &= \sum_{\omega \in A} p_w \\
   			&= \sup\{\sum_{\omega \in A_0} p_w | \text{ finite} A_0 \subset A \} \\
   			&= \sup\{\sum_{i = 1}^n \sum_{\omega \in A_i} p_w | \text{ finite} A_i \subset A_0 \subset A  \} \quad \text{(after relabelling)}   \\
   			&= \sum_{i = 1}^{\infty} \sum_{\omega \in A_i} p_w \quad \text{(definition of series as limit of the (increasing) sequence of partial sums)} \\
   			&= \sum_{i = 1}^{\infty} \bbP(A_i).
   \end{align*}
   Note the suprema above are finite, since the indexed sum over $\Omega$ is 1 and we are summing subsets of omega. Hence $\bbP$ is a measure on $(\Omega, 2^{\Omega})$.
\end{solution}

\begin{solution}[1.14]{Vrămuleț, A. Ș}\\
Let $x \in A$. Then $x$ is an interior point of A. So there exists $r > 0$ such that 
\[
B(x,r) \subset A, \text{ where } B(x,r) = (x-r,x+r).
\]
Since $\bbQ$ is dense in $\bbR$, then there exist rationals $p$ and $q$ such that
\[
p \in (x-r,x) \text{ and } q \in (x,x+p).
\]
So $(p,q) \subset (x-r,x+r) \subset A$ and $x \in (p,q)$. Note there are countably many such intervals (since $\bbQ$ is countable, so is $\bbQ \times \bbQ$). So A is the union of countably many intervals with rational endpoints of the form $(p,q)$.
\end{solution}

\begin{solution}[1.15]{Vrămuleț, A. Ș}\\
Note from before that an indexed sum of positive reals is finite if and only if there are countably many nonzero reals (in the sum). So in the case that at least one of $I$ and $J$ is uncountable, we actually have countable sums. \\

\noindent So the sums can be split as uncountable sum of zeros plus countable sum of positive reals. So in all cases the sums reduce to summing over countable sets.
It's known that the result holds (by considering (absolute) convergence of the series)). Note the result also holds when the series converges to $\infty$ (over the extended real line).
\end{solution}

\chapter{Solutions: Measurable sets and \texorpdfstring{$\sigma$}{}-algebras}

\begin{solution}[2.1]{Beerens, L.}
    Let $(\Omega, \mathcal{A}_0)$ be a measurable space, where $\Omega\neq\varnothing$. Assume that $\mathcal{A}_0$ has an odd number of elements. Choose $n\in\mathbb{N}$ such that $|\mathcal{A}_0|=2n+1$. Choose some set $A\in \mathcal{A}_0$. Since $\mathcal{A}_0$ is a $\sigma$-algebra, we know that $A^c\in\mathcal{A}_0$. Let $\mathcal{A}_1 = \mathcal{A}\setminus\{A, A^c\}$.
    For all elements $A\in\mathcal{A}_1$, the complement is clearly still in $\mathcal{A}_1$.
    We can repeat the process by noticing that
    $$
        \forall_{A\in\mathcal{A}_k}: A^c\in\mathcal{A}_k,
    $$
    for all $k\leq n$.
    For $k=0$, it is true since we know that $\mathcal{A}_0$ is a $\sigma$-algebra. Assume that for some $k\in\mathbb{N}\cap[0,n)$ we have
    $$
        \forall_{A\in\mathcal{A}_k}: A^c\in\mathcal{A}_k.
    $$
    Note that $|\mathcal{A}_k|=2(n-k)+1\geq 3$, since $k\leq n -1$. We can thus create $\mathcal{A}_{k+1}$ using the method of picking some set $A\in\mathcal{
    A}_{k}$ and defining $\mathcal{A}_{k+1} = \mathcal{A}_k\setminus\{A, A^c\}$. Let $B\in\mathcal{A}_{k+1}\subset\mathcal{A}_0$. By the fact that $\mathcal{A}_0$ is a $\sigma$-algebra, we know that $B^c\in\mathcal{A_0}$. Clearly, $B^c$ has not been removed in any of the $k+1$ steps, since that would imply that $B$ was removed too. Therefore, $B^c\in\mathcal{A}_{k+1}$. By induction, we find that
    $$
        \forall_{A\in\mathcal{A}_n}: A^c\in\mathcal{A}_n.
    $$
    However, $|\mathcal{A}_n|=1$. Let $A\in\mathcal{A}_n$. Since we assumed that $\Omega\neq \varnothing$, we know that $A^c\neq A$, which leads to a contradiction. Therefore, we can conclude that $\mathcal{A}_0$ has an even amount of elements.
\end{solution}

\begin{solution}[2.2]{Castella, A.}
    By the definition of a $\sigma$-algebra, all countable unions of sets in $\mathcal{F}$ are also in $\mathcal{F}$. Thus let create a sequence $(A_n)_{n\in\mathbb{N}}$ such that $A_1 = A$ and $A_i = B$ for all $i > 1$. Clearly
    $$
        \bigcup_{i=1}^\infty A_i = A\cup B \in \mathcal{F}.
    $$
    Since $A$ and $B$ are arbitrary, this must hold for all pairs of sets in $\mathcal{F}$. From the fact that $\mathcal{F}$ is a $\sigma$-algebra we find that $A^c,B^c \in \mathcal{F}$. As we have proven that unions of pairs are in $\mathcal{F}$, we find that
    $$
        (A^c \cup B^c)^c = A\cap B \in \mathcal{F}.
    $$
    This again holds for all possible pairs of sets in $\mathcal{F}$. Now let us note that
    $$
        A\setminus B = A\cap B^c.
    $$
    It is clear from the properties that we have already proven and the complementation property of $\sigma$-algebras, we can conclude that
    $$
        A\setminus B \in \mathcal{F}.
    $$
    The final set, $A\Delta B$ follows by definition. The set is defined by $(A\cup B)\setminus (A\cap B)$. It is clear that this is a composition of the properties we have already proven. Since the sets $A$ and $B$ were arbitrary, we can conclude that
    $$
        A\Delta B \in \mathcal{F}.
    $$
\end{solution}

\begin{solution}[2.4]{Castella, A.}
    A $\sigma$-algebra $\mathcal{G}$ is defined as a set of subsets such that
    \begin{itemize}
        \item $E \in \mathcal{G}$,
        \item for all infinite sequences $\{A_i\}$ in $\mathcal{G}$ the union $\bigcup_{i=1}^\infty A_i$ also belongs to the set $\mathcal{G}$,
        \item for all sets $A$ in $\mathcal{G}$, the set $A^c$ belongs to $\mathcal{G}$.
    \end{itemize}
    Therefore, we begin by verifying that $\Omega \in \mathcal{F}$. From the definition of $f$ it directly follows that
    $$
        f^{-1}(E) = \Omega.
    $$
    Since $\mathcal{G}$ is a $\sigma$-algebra, we find that $E \in \mathcal{G}$. Therefore, we conclude that $\Omega \in \mathcal{F}$.

    We now verify the second condition. Let us take an arbitrary infinite sequence $\{A_i\}$ in $\mathcal{F}$. By the definition of $\mathcal{F}$ we know that for all $A_i$, there exists $B_i \in \mathcal{G}$ such that $A_i = f^{-1}(B_i)$. In order to prove the condition, we notice that
    $$
        \bigcup_{i=1}^\infty B_i \in \mathcal{G}.
    $$

    Before we continue with the proof, we need to prove an intermediary. Let us take some family of sets $\{C_\alpha\}_{\alpha\in\mathcal{I}}$ where $\mathcal{I}$ is an index set. We will prove that for some function $g : \Omega \rightarrow E$ we have
    $$
        g^{-1}\left(\bigcup_{\alpha\in\mathcal{I}}C_\alpha\right) = \bigcup_{\alpha\in\mathcal{I}}g^{-1}(C_\alpha).
    $$
    We begin by showing that $g^{-1}\left(\bigcup_{\alpha\in\mathcal{I}}C_\alpha\right) \subset \bigcup_{\alpha\in\mathcal{I}}g^{-1}(C_\alpha)$. Let us take an arbitrary $a \in g^{-1}\left(\bigcup_{\alpha\in\mathcal{I}}C_\alpha\right)$. Then we find that there exists some $b \in \bigcup_{\alpha\in\mathcal{I}}C_\alpha$ such that $g(a) = b$. This implies that there exists an $\alpha \in \mathcal{I}$ such that $b \in C_\alpha$. From this we find that
    $$
        a \in g^{-1}(C_\alpha) \subset \bigcup_{\alpha\in\mathcal{I}}g^{-1}(C_\alpha).
    $$
    Since $a$ was chosen arbitrarily, we can conclude that
    $$
        g^{-1}\left(\bigcup_{\alpha\in\mathcal{I}}C_\alpha\right) \subset \bigcup_{\alpha\in\mathcal{I}}g^{-1}(C_\alpha).
    $$
    Now we show that the converse is also true. Let us take an arbitrary $a' \in \bigcup_{\alpha\in\mathcal{I}}g^{-1}(C_\alpha)$. Then we find that there must exist some $\alpha' \in \mathcal{I}$ such that $a' \in g^{-1}(C_{\alpha'})$. We can now choose some $b' \in C_{\alpha'}$ such that $g(a') = b'$. It is clear that $b' \in \bigcup_{\alpha\in\mathcal{I}}C_\alpha$ as well. Thus we know that
    $$
        a' \in g^{-1}\left(\bigcup_{\alpha\in\mathcal{I}}C_\alpha\right).
    $$
    Since again, our choice of $a'$ was arbitrary, we can conclude that
    $$
        \bigcup_{\alpha\in\mathcal{I}}g^{-1}(C_\alpha) \subset g^{-1}\left(\bigcup_{\alpha\in\mathcal{I}}C_\alpha\right).
    $$
    These two inequalities clearly imply that
    $$
        g^{-1}\left(\bigcup_{\alpha\in\mathcal{I}}C_\alpha\right) = \bigcup_{\alpha\in\mathcal{I}}g^{-1}(C_\alpha).
    $$

    Using the result that we have just proven we can continue with the proof. For the infinite sequence $\{A_i\}$ we find that
    $$
        f^{-1}\left(\bigcup_{i=1}^\infty B_i\right) = \bigcup_{i=1}^\infty f^{-1}(B_i) = \bigcup_{i=1}^\infty A_i.
    $$
    By the definition of our set $\mathcal{F}$ we now find that
    $$
        \bigcup_{i=1}^\infty A_i \in \mathcal{F}.
    $$
    Therefore, we conclude that the second condition holds for $\mathcal{F}$ as well.

    We now prove the third and final condition. Let us take $A \in \mathcal{F}$. By the definition of $\mathcal{F}$ we find that there exists $B \in \mathcal{G}$ such that
    $$
        A = f^{-1}(B).
    $$
    By the definition of a $\sigma$-algebra we know that $B^c \in \mathcal{G}$. Additionally, we know that
    $$
        f^{-1}(B^c) = f^{-1}(B)^c = A^c.
    $$
    By using these two facts and the definition of the set $\mathcal{F}$ again, we find that
    $$
        A^c \in \mathcal{F}.
    $$
    This proves that the third condition holds as well.

    Since all of the required conditions hold, we can come to the final conclusion that $\mathcal{F}$ is indeed a $\sigma$-algebra.
\end{solution}

\begin{solution}[2.5]{Castella, A.}
\begin{itemize}
    \item Before we begin with the proof, we will prove the intermediary that if $A,B \in \mathcal{F}$, then $A \setminus B \in \mathcal{F}$. We begin by noting that
    $$
        A \setminus B = A \cap (B^c).
    $$
    From this it becomes very clear that the statement is true. We know by the definition of a $\sigma$-algebra that $B \in \mathcal{F}$ implies $B^c \in \mathcal{F}$. Additionally, we know that intersections of infinite sequences also belong to the same $\sigma$-algebra. We take the sequence $A_1 = A$, $A_i = B^c$ for $i \in \mathbb{N}\setminus\{1\}$. With this we find that $A \cap B^c \in \mathcal{F}$ and therefore
    $$
        A\setminus B \in \mathcal{F}.
    $$
    We now proceed to showing that for all infinite sequences $(A_n)_{n\in\mathbb{N}}$, there exits a mutually disjoint sequence who's union is equal to $\cup_{n\in\mathbb{N}}A_n$. We define the sequence $(E_n)_{n\in\mathbb{N}}$ such that
    $$
        E_n = A_n \setminus \bigcup_{i=1}^{n-1}A_i.
    $$
    By the intermediary and since infinite unions, and by the same argument as in the intermediary, also finite unions are contained in the $\sigma$-algebra, it is easy to see that
    $$
        E_n \in \mathcal{F}
    $$
    for all $n\in\mathbb{N}$. It is clear from the definition of the sequence that it is mutually disjoint and that its union is equal to the union of $(A_n)_{n\in\mathbb{N}}$.
    \item We take the same arbitrary sequence $(A_n)_{n\in\mathbb{N}}$ as in the previous item. We define the sequence $(F_n)_{n\in\mathbb{N}}$ by
    $$
        F_n = A_n \cup \left(\bigcup_{i=1}^{n-1}A_i\right) = \bigcup_{i=1}^n A_i.
    $$
    We first note that it is clear from the definition that this is an increasing sequence, as each element is the union of $A_n$ and all of its predecessors. As mentioned in the previous item, infinite unions are contained in the $\sigma$-algebra as well as finite unions. Therefore we know that
    $$
        F_n \in \mathcal{F}
    $$
    for all $n \in \mathbb{N}$. From the definition of a union we also know that
    $$
        \bigcup_{i=1}^\infty \left(\bigcup_{j=1}^i A_j\right) = \bigcup_{i=1}^\infty A_n.
    $$
    Thus we have now proven that the sequence $(F_n)_{n\in\mathbb{N}}$ is such that $F_n \subset F_{n+1}$ and $\cup_{n\in\mathbb{N}}A_n = \cup_{n\in\mathbb{N}}F_n$.
\end{itemize}
\end{solution}


\begin{solution}[2.13]{Beerens, L.}
    Let $\Omega_1$ and $\Omega_2$ be two non-empty sets, and let $\mathcal{F}_1$ and $\mathcal{F}_2$ be $\sigma$-algebras on $\Omega_1$ and $\Omega_2$ respectively. We consider the product $\sigma$-algebra on $\Omega_1\times\Omega_2$ given by
    $$
        \mathcal{F}_1\otimes\mathcal{F}_2:=\sigma(\{ A_1\times A_2: A_1\in\mathcal{F}_1, A_2\in\mathcal{F}_2 \}).
    $$
    Suppose that $\mathcal{F}_1$ is generated by $\mathcal{A}_1$ and $\mathcal{F}_2$ is generated by $\mathcal{A}_2$. Let $\mathcal{F}:=\mathcal{F}_1\otimes\mathcal{F}_2$ and
    $$
        \mathcal{S}:=\sigma(\{ A_1\times A_2: A_1\in\mathcal{A}_1, A_2\in\mathcal{A}_2 \}).
    $$
    Let
    $$
        A\in\{ A_1\times A_2: A_1\in\mathcal{A}_1, A_2\in\mathcal{A}_2 \}.
    $$
    Then $A=A_1\times A_2$ for some $A_1\in\mathcal{A}_1$ and $A_2\in\mathcal{A}_2$. Since $\mathcal{A}_1\subset\mathcal{F}_1$ and $\mathcal{A}_2\subset\mathcal{F}_2$, we find that
    $$
        A\in\{ A_1\times A_2: A_1\in\mathcal{F}_1, A_2\in\mathcal{F}_2 \}.
    $$
    Therefore,
    $$
        \{ A_1\times A_2: A_1\in\mathcal{A}_1, A_2\in\mathcal{A}_2 \}\subset\{ A_1\times A_2: A_1\in\mathcal{F}_1, A_2\in\mathcal{F}_2 \},
    $$
    from which it follows that $\mathcal{S}\subset\mathcal{F}$ (By \textbf{Problem 2.12}).

    To prove the converse, suppose that $F\in\{ A_1\times A_2: A_1\in\mathcal{F}_1, A_2\in\mathcal{F}_2 \}$. Then there exist $A_1\in\mathcal{F}_1$ and $A_2\in\mathcal{F}_2$ such that $F = A_1\times A_2 = (A_1\times\Omega_2)\cap(\Omega_1\times A_2)$. By definition
    $$
        \forall_{A\in\mathcal{A}_1}: A\times\Omega_2\in \mathcal{S}.
    $$
    Thus we find that
    $$
        \sigma(\mathcal{A}_1)\times \{\Omega_2\} = \sigma(\mathcal{A}_1\times \{\Omega_2\})\subset \mathcal{S}.
    $$
    Since $A_1\in\mathcal{F}_1 = \sigma(\mathcal{A}_1)$, it now follows that $A_1\times\Omega_2\in \mathcal{S}$. Analogously, $\Omega_1\times A_2\in \mathcal{S}$. Since $\mathcal{S}$ is a $\sigma$-algebra, it follows that $F\in \mathcal{S}$. Thus, $\mathcal{F}\subset\mathcal{S}$, from which we can conclude that $\mathcal{F}=\mathcal{S}$, as was to be shown.
\end{solution}

\begin{solution}[2.14]{Castella, A.}
    In order to prove equality of the sets, we will prove that they are both subsets of one another.
    \begin{itemize}
        \item As we know, the set $\mathcal{B}_{\mathbb{R}^2}$ is generated by the $\pi$-system of open rectangles. Let us assume that $\mathcal{A}$ is the set of open rectangles in $\mathbb{R}^2$. Let us take some arbitrary $A \in \mathcal{A}$, then there exists $a,b,c,d \in \mathbb{R}$ such that $a < b$, $c < d$, and $A = (a,b)\times(c,d)$. Since the Borel set $\mathcal{B}_\mathbb{R}$ is generated by the set of open intervals in $\mathbb{R}$, we find that $(a,b), (c,d) \in \mathcal{B}_\mathbb{R}$. From this we immediately find that $(a,b)\times(c,d) \in \mathcal{B}_\mathbb{R}\times\mathcal{B}_\mathbb{R}$. Since $\mathcal{B}_\mathbb{R}\otimes\mathcal{B}_\mathbb{R}$ is the $\sigma$-algebra generated by the cross product $\mathcal{B}_\mathbb{R}\times\mathcal{B}_\mathbb{R}$ we find that
        $$
            \mathcal{B}_{\mathbb{R}^2} \subset \mathcal{B}_\mathbb{R}\otimes\mathcal{B}_\mathbb{R}.
        $$
        \item We now prove the converse. Let us begin by defining the projections $\pi_1$ and $\pi_2$ as
        $$
            \pi_1(x,y) = x
        $$
        and
        $$
            \pi_2(x,y) = y.
        $$
        We will show that these two functions are measurable with respect to $\cB_{\bbR^2}$ and $\cB_{\bbR}$. Let us take an arbitrary $t \in \mathbb{R}$. We will show that the set
        $$
            A = \{(x\times y) \in \mathbb{R}^2 : \pi_1(x,y) < t\}
        $$
        is a Borel set. We find that $\pi_1(x,y) < t$ if and only if $x < t$. Thus we find that
        $$
            A = (-\infty, t) \times \mathbb{R}.
        $$
        Let us define the set $B_n$ as
        $$
            B_n = (-n,t) \times (-n,n)
        $$
        Clearly, for all $n\in\mathbb{N}$, the set $B_n$ is an open rectangle and therefore $B_n \in \cB_{\bbR^2}$. We also find that
        $$
            \bigcup_{i=1}^\infty B_n = (-\infty,t)\times \mathbb{R} = A.
        $$
        Since $\cB_{\bbR^2}$ is a $\sigma$-algebra, it contains all countable unions of its sets. Therefore, we find that
        $$
            \bigcup_{i=1}^\infty B_n = A \in \cB_{\bbR^2}.
        $$
        Thus the set $A$ is indeed a Borel measurable set. Since $t$ was chosen arbitrarily, this holds for all $t \in \mathbb{R}$. Thus $\pi_1$ is a measurable function. The proof is analogous for $\pi_2$. Since both of these functions are measurable, we know that the preimage of a set in $\cB_{\bbR}$ is in $\cB_{\bbR^2}$. Therefore, for all $A \in \cB_{\bbR}$, we find that
        $$
            A \times \mathbb{R} \in \cB_{\bbR^2},
        $$
        by the measurability of $\pi_1$ and
        $$
            \mathbb{R} \times A \in \cB_{\bbR^2},
        $$
        by the measurability of $\pi_2$. Let us take arbitrary $A,B \in \cB_{\bbR}$. By our previous result and since $\cB_{\bbR^2}$ is a $\sigma$-algebra and thus contains countable intersections, we know that
        $$
            (A\times\mathbb{R})\cap(\mathbb{R}\times B) = A\times B \in \cB_{\bbR^2}.
        $$
        By the fact that $A$ and $B$ were chosen arbitrarily we find that
        $$
            \cB_{\bbR} \times \cB_{\bbR} \subset \cB_{\bbR^2}.
        $$
        Since $\cB_{\bbR}\otimes\cB_{\bbR}$ is the $\sigma$-algebra generated by the Cartesian product of $\cB_{\bbR}$ with itself, we can conclude that
        $$
            \cB_{\bbR}\otimes\cB_{\bbR} \subset \cB_{\bbR^2}.
        $$
    \end{itemize}
    Combining both of the results, we arrive at the final conclusion that
    $$
        \mathcal{B}_{\mathbb{R}^2} = \mathcal{B}_\mathbb{R}\otimes\mathcal{B}_\mathbb{R}.
    $$
\end{solution} 

\chapter{Solutions: Measures}

\begin{solution}[3.11]{Beerens, L.}
    \begin{itemize}
        \item Let $x_1,x_2\in\mathbb{R}$ with $x_1<x_2$. Then
        $$
            F(x_2) = \mu((-\infty,x_1]\cup (x_1,x_2]) = \mu((-\infty, x_1)) + \mu((x_1,x_2])\geq\mu((-\infty, x_1)) = F(x_1).
        $$
        Therefore, $F$ is non-decreasing.
        
        \item Consider $\lim_{x\downarrow x_0}F(x)$. This can be written as 
        $$
            \lim_{x\downarrow x_0}\mu((-\infty,x]).
        $$
        We know that this limit exists, so we can look at a sequence that converges to $x_0$ from above instead. Let $(x_n)$ be a strictly decreasing sequence in $\mathbb{R}$ such that $x_n\rightarrow x_0$. For all $n\in\mathbb{N}$, let $A_n:=(-\infty,x_n]$. Then
        $$
            \lim_{x\downarrow x_0}\mu((-\infty,x]) = \lim_{n\rightarrow \infty}\mu((-\infty,x_n]) = \lim_{n\rightarrow \infty}\mu(A_n).
        $$
        Since $A_n$ is a decreasing sequence of measurable sets such that $\mu(A_1)$ is finite, we find that
        $$
            \lim_{n\rightarrow \infty}\mu(A_n) = \mu\left( \bigcap_{i=1}^\infty A_i \right) = \mu((-\infty, x_0]),
        $$
        which completes the proof.
        
        \item We shall turn towards $\lim_{x\rightarrow -\infty}F(x)$. We know that this limit exists, so we can look at a sequence that diverges to $-\infty$ instead. Let the sequence $(x_n)$ be defined by $x_n = -n$. For all $n\in\mathbb{N}$, let $A_n:=(-\infty,x_n]$. Then
        $$
            \lim_{x\downarrow x_0}\mu((-\infty,x]) = \lim_{n\rightarrow \infty}\mu((-\infty,x_n]) = \lim_{n\rightarrow \infty}\mu(A_n).
        $$
        Since $A_n$ is a decreasing sequence of measurable sets such that $\mu(A_1)$ is finite, we find that
        $$
            \lim_{n\rightarrow \infty}\mu(A_n) = \mu\left( \bigcap_{i=1}^\infty A_i \right) = \mu(\varnothing) = 0,
        $$
        which completes the proof.
        
        \item Finally, we consider $\lim_{x\rightarrow \infty}F(x)$. We know that this limit exists, so we can look at a sequence that diverges to $\infty$ instead. Let $(x_n)$ be a sequence in $\mathbb{R}$ defined by $x_n = n$. For all $n\in\mathbb{N}$, let $A_n:=(-\infty,x_n]$. Then
        $$
            \lim_{x\rightarrow \infty}F(x) = \lim_{n\rightarrow \infty}F(x_n).
        $$
        Since $A_n$ is an increasing sequence of measurable sets, we find that
        $$
            \lim_{n\rightarrow \infty}\mu(A_n) = \mu\left( \bigcup_{i=1}^\infty A_i \right) = \mu(\mathbb{R}),
        $$
        which completes the proof.
    \end{itemize}
\end{solution}

\begin{solution}[3.14]{Beerens, L.}
    Let $(\Omega,\mathcal{F})$ be a measure space. Let $\mu:\mathcal{F}\rightarrow[0,\infty]$ be a set function which is finitely additive and such that $\mu(\varnothing) = 0$. If $\mu$ is a measure, then it is continuous from below, since that is a property of measures. Conversely, suppose that $\mu$ is continuous from below. Note that $\mu$ assigns to each set $A\in\mathcal{F}$ a nonnegative extended real number $\mu(A)$. It was already assumed that $\mu(\varnothing) =0$, so we will proceed by proving that $\mu$ is $\sigma$-additive. Let $A_1, A_2,\hdots$ be a sequence of mutually disjoint elements of $\mathcal{F}$. For all $n\in\mathcal{N}$, let 
    $$
        B_n = \bigcup_{i=1}^n A_i.
    $$
    Since $\mathcal{F}$ is a $\sigma$-algebra, we find that all $B_n$ are sets in $\mathcal{F}$.
    From finite additivity, we find that for all $n\in\mathbb{N}$ we have
    $$
        \mu(B_n) = \mu\left( \bigcup_{i=1}^n A_i \right) = \sum_{i=1}^n \mu(A_i). 
    $$
    Note that $(B_n)$ is an increasing sequence of measurable sets in $\mathcal{F}$. By continuity from below it follows that
    $$
        \mu\left( \bigcup_{i=1}^\infty B_i \right) = \lim_{n\rightarrow\infty}\mu(B_n) = \sum_{i=1}^\infty\mu(A_i).
    $$
    Therefore,
    $$
        \mu\left( \bigcup_{i=1}^\infty A_i \right) = \mu\left( \bigcup_{i=1}^\infty B_i \right) = \sum_{i=1}^\infty\mu(A_i).
    $$
    Thus $\mu$ is $\sigma$-additive and we can conclude that $\mu$ is a measure.
\end{solution}

\begin{solution}[3.21]{Beerens, L.}
    Let $F:\mathbb{R}\rightarrow\mathbb{R}$ be a non-decreasing, right-continuous function. Let $\nu_F$ be the unique measure on $(\mathbb{R},\mathcal{B}_\mathbb{R})$ such that $\nu_F((a,b])=F(b)-F(a)$ for all $a<b$. For all $i\in\mathbb{Z}$, let
    $$
        A_i:= (i,i+1]
    $$
    and let $\mathcal{A}$ be the collection of all these sets $A_i$. Note that $\mathcal{A}$ is a countable cover of $\mathbb{R}$. By the assumption we know that for all $i\in\mathbb{Z}$ we have
    $$
        \nu(A_i)  = F(i+1) - F(i).
    $$
    Since $F$ takes values in $\mathbb{R}$, we find that $\nu(A_i)$ is finite for all $i\in\mathbb{Z}$. Therefore, $\mathcal{A}$ is a countable cover of $\mathbb{R}$ that consists of sets with finite measure. Thus we conclude that $\mu$ is $\sigma$-finite.
\end{solution}


\end{document}