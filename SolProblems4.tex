\begin{solution}[5.3]{Kempen, S.F.M.}

    \noindent "$\Rightarrow$" Let $s,t\in \mathbb{R}$ and define $A = \{X \geq s\}, B = \{Y \geq t\}$. We have
    $$ A = \{X \geq s\} = \{\omega\in \Omega: X(\omega) \geq s\} = X^{-1}([s,\infty)). $$
    By definition $\sigma(X)$ is the smallest $\sigma$-algebra such that $X^{-1}(B)\in \sigma(X)$ for all $B\in \mathcal{B}_\mathbb{R}$ and $[s,\infty)\in \mathcal{B}_\mathbb{R}$ so surely $A\in \sigma(X)$. In a similar way it can be shown that $B\in \sigma(Y)$. Now
    $$ \mathbb{P}(X\geq s, Y\geq t) = \mathbb{P}(\{X\geq s\} \cap \{Y\geq t\}) = \mathbb{P}(A\cap B) $$
    and by our assumption
    $$ \mathbb{P}(A\cap B)  = \mathbb{P}(A) \mathbb{P}(B) = \mathbb{P}(\{X\geq s\})\cdot\mathbb{P}(\{Y\geq t\}) = \mathbb{P}(X\geq s)\cdot\mathbb{P}(Y\geq t). $$
    
    
    \noindent "$\Leftarrow$" For the first part of the proof, let us define
    \begin{align*}
        \Lambda_s &= \{ \ B\in \sigma(Y): \mathbb{P}(X\geq s, B) = \mathbb{P}(X\geq s)\mathbb{P}(B)\ \}.
    \end{align*}
    $\Lambda_s$ is a collection of sets in $\sigma(Y)$ so $\Lambda_s \subseteq \sigma(Y)$. Furthermore we can show that $\Lambda_s$ is a $\lambda$-system.
    \begin{enumerate}
        \item We have $\Omega \in \Lambda_s$ since 
        \begin{align*}
            \mathbb{P}(X\geq s, \Omega) &= \mathbb{P}(\{X\geq s\} \cap \Omega) = \mathbb{P}(\{\omega\in \Omega: X(\omega) \geq s\}\cap \Omega) \\
            &= \mathbb{P}(\{\omega\in \Omega: X(\omega) \geq s\}) \\
            &= \mathbb{P}(X\geq s) \cdot 1 = \mathbb{P}(X\geq s)\mathbb{P}(\Omega).
        \end{align*}
        \item Let $A \in \Lambda_s$ then 
        \begin{align*}
            \mathbb{P}(X\geq s, \Omega\setminus A) &= \mathbb{P}(\{X\geq s\} \cap \Omega\setminus A) \\
            &= \mathbb{P}(\{\omega\in \Omega: X(\omega)\geq s\} \cap \Omega\setminus A)\\
            &= \mathbb{P}(\{\omega\in \Omega\setminus A: X(\omega)\geq s\})\\
            &= \mathbb{P}(X\geq s) - \mathbb{P}(\{X\geq s\} \cap A)\\
            &= \mathbb{P}(X\geq s)(1-\mathbb{P}(A)) \quad (\textrm{since }A\in\Lambda_s)\\
            &= \mathbb{P}(X\geq s)\mathbb{P}(\Omega\setminus A). 
        \end{align*}
        \item Let $(A_i) \subset \Lambda_s$ be a sequence of mutually disjoint sets then by  the $\sigma$-additivity of the measure we have
        \begin{align*}
            \mathbb{P}(X\geq s, \bigcup_{i=1}^\infty A_i) &= \mathbb{P}(\{X\geq s\}\cap \bigcup_{i=1}^\infty A_i)\\
            &= \mathbb{P}(\{\omega\in \left(\Omega\cap \bigcup_{i=1}^\infty A_i\right): X(\omega)\geq s\} )\\
            &= \sum_{i=1}^\infty \mathbb{P}( \{\omega\in \Omega\cap A_i: X(\omega)\geq s\} )\\
            &= \sum_{i=1}^\infty \mathbb{P}(\{X\geq s\} \cap A_i)\\
            &= \sum_{i=1}^\infty \mathbb{P}(\{X\geq s\}) \mathbb{P}(A_i)\\
            &= \mathbb{P}(X\geq s) \mathbb{P}(\bigcup_{i=1}^\infty A_i).
        \end{align*}
    \end{enumerate}
    So we conclude that $\Lambda_s$ is a $\lambda$-system. By assumption, we also have $\{Y\geq t\}\subseteq \Lambda_s$ for all $t\in \mathbb{R}$.
    
    \noindent Next we define 
    $$ \mathcal{E} = \{ \{Y\geq t\},t\in \mathbb{R}\}$$
    and show that this is a $\pi$-system. Let $s,r\in \mathbb{R}$ then $\{Y\geq s\},\{Y\geq r\}\in \mathcal{E}$ and
    $$ \{Y\geq s\} \cap \{Y\geq r\} = \{Y\geq \min(s,r)\} \in \mathcal{E},$$
    since the minimum of two real numbers is just a real number. Thus $\mathcal{E}$ is a $\pi$-system.
    
    \noindent Note $\sigma(Y) \subseteq \sigma(\mathcal{E})$ by definition of $\sigma(Y)$ and next to that $\mathcal{E} \subseteq \Lambda_s$ by our assumption. We can now apply the $\pi$-$\lambda$ theorem, which tells us
    $$ \sigma(Y) \subseteq \sigma(\mathcal{E}) \stackrel{\pi-\lambda}{\subseteq} \Lambda_s \subseteq \sigma(Y), $$
    so in fact we have $\Lambda_s = \sigma(Y)$. Thus we can conclude that $\mathbb{P}(X\geq s, B) = \mathbb{P}(X\geq s)\mathbb{P}(B)$ holds for all $B\in \sigma(Y)$.
    
    \noindent For the second part of the proof, define
    $$ \mathcal{H} = \{A\in \sigma(X): \mathbb{P}(A\cap B) = \mathbb{P}(A)\mathbb{P}(B) \ \forall  B\in \sigma(Y) \}. $$
    We will show that $\mathcal{H}$ is a $\lambda$-system. Let $B\in \sigma(Y)$.
    \begin{enumerate}
        \item We have $\Omega \in \mathcal{H}$ since $\mathbb{P}(\Omega\cap B) = \mathbb{P}(B) = \mathbb{P}(B)\mathbb{P}(\Omega)$.
        \item Let $A \in \mathcal{H}$ then 
        \begin{align*}
            \mathbb{P}(B \cap (\Omega\setminus A) ) &= \mathbb{P}(B\setminus (A\cap B))\\
            &= \mathbb{P}(B) - \mathbb{P}(A\cap B) \quad \textrm{(by exclusion)}\\
            &= \mathbb{P}(B)(1-\mathbb{P}(A)) \quad \textrm{(since } A\in \mathcal{H})\\
            &= \mathbb{P}(B)\mathbb{P}(\Omega \setminus A).
        \end{align*}
        \item Let $(A_i) \subset \mathcal{H}$ be a sequence of mutually disjoint sets then by the $\sigma$-additivity of the measure $\mathbb{P}(\bigcup_{i=1}^\infty A_i) = \sum_{i=1}^\infty \mathbb{P}(A_i)$ and
        \begin{align*}
            \mathbb{P}(B\cap \left(\bigcup_{i=1}^\infty A_i\right)) &= \mathbb{P}(\bigcup_{i=1}^\infty (B \cap A_i))\\
            &= \sum_{i=1}^\infty \mathbb{P}(B\cap A_i) \quad (\sigma\textrm{-add.})\\
            &= \sum_{i=1}^\infty \mathbb{P}(B) \mathbb{P}(A_i)\\
            &= \mathbb{P}(B)\mathbb{P}(\bigcup_{i=1}^\infty A_i).
        \end{align*}
    \end{enumerate}
    So $\mathcal{H}$ is a $\lambda$-system.

    \noindent Let $\mathcal{G} = \{\{X\geq s\}: s\in \mathbb{R}\}$. It can be shown that $\mathcal{G}$ is a $\pi$-system in a similar way as we did for $\mathcal{E}$ and, as before, $\sigma(\mathcal{G}) \subseteq \sigma(X)$ (by definition of $\sigma(X)$ and $\mathcal{G} \in \mathcal{H}$ (by assumption). Applying the $\pi$-$\lambda$ theorem we obtain
    \begin{align*}
        \sigma(X) \subseteq \sigma(\mathcal{G}) \stackrel{\pi-\lambda}{\subseteq} \mathcal{H} \subseteq \sigma(X),
    \end{align*}
    which implies $\mathcal{H} = \sigma(X)$. Thus now we have
    $$ \mathbb{P}(A\cap B) = \mathbb{P}(A)\mathbb{P}(B) \ \forall A\in \sigma(X), B\in \sigma(Y).  $$

\end{solution}

\begin{solution}[5.9]{Beerens, L.}
    Let $(\Omega, \mathcal{F})$ be a measurable space and $(f_n)$ be a sequence of $\mathcal{F}$-measurable functions. 
    We know that $\limsup_{n\rightarrow\infty}f_n$ and $\liminf_{n\rightarrow\infty}f_n$ are $\mathcal{F}$-measurable functions. For all $\omega\in\Omega$ we have 
    $$
        \lim_{n\rightarrow\infty}f_n(\omega) \text{ exists} \iff \limsup_{n\rightarrow\infty}f_n(\omega)=\liminf_{n\rightarrow\infty}f_n(\omega).
    $$
    From problem 5.8 it now follows that
    $$
        \{\omega \in \Omega : \lim_{n\rightarrow\infty}f_n(\omega) \text{ exists}\} = \{\omega \in \Omega :\limsup_{n\rightarrow\infty}f_n(\omega)=\liminf_{n\rightarrow\infty}f_n(\omega) \}
    $$
    is measurable.
\end{solution}

\begin{solution}[5.19]{Beerens, L.}
    Suppose $f:\mathbb{R}\rightarrow\mathbb{R}$ is a non-negative function such that $\sqrt{f}$ is measurable. It is known that the product of two measurable functions is measurable. Therefore, $\sqrt{f}\cdot\sqrt{f}=f$ is measurable. Thus, there does not exist such non-measurable $f$. Since it was not proven in the notes that the product of two measurable functions is measurable, we shall give a quick proof here. 
    
    Let $f,g:(\Omega,\mathcal{F})\rightarrow \mathbb{R}$ be $\mathcal{F}$-measurable. First, we show that $f^2$ is measurable. For all $t\in\mathbb{R}$, we have
    $$
        \{ \omega\in\Omega:f^2(\omega)\geq t \} = \{ \omega\in\Omega:|f(\omega)|\geq\sqrt{\max\{0,t\}} \},
    $$
    which is measurable by the measurability of $|f|$. Therefore, $f^2$ is measurable. In the same way, the square of any measurable function is measurable, including $g^2$ and $(f+g)^2$. But $(f+g)^2 = f^2 + g^2 +2fg$, so $fg = 1/2 ((f+g)^2 - f^2 - g^2)$ and hence is measurable.
\end{solution}

\begin{solution}[5.20]{Beerens, L.}
    Suppose $f:(\mathbb{R},\overline{\mathcal{B}}^\mathcal{L})\rightarrow(\mathbb{R},\mathcal{B}_\mathbb{R})$ is continuous almost everywhere. Then $f$ is continuous at $\omega$ for all $\omega\in\mathbb{R}\setminus U$ and not continuous at $\omega$ for all $\omega\in U$, for some set $U\in \overline{\mathcal{B}}^\mathcal{L}$ such that $\mathcal{L}^1(U)=0$. Let $A=\mathbb{R}\setminus U$. Then $f|_A$ is continuous and therefore measurable. Since $U$ has measure zero, we know that all its subsets are measurable, as the Lebesgue measure is a complete measure. Therefore, $f|_U$ is a measurable function. Now, we shall prove that $f$ is measurable. Let $B\in\mathcal{B}_\mathbb{R}$. Since $f|_A$ and $f|_U$ are measurable, we find that there exists $C,D\in\overline{\mathcal{B}}^\mathcal{L}$ such that
    $$
        f|_A^{-1}(B) = C\cap A
    $$ 
    and
    $$
        f|_U^{-1}(B) = D\cap U.
    $$
    Therefore,
    $$
        f^{-1}(B) = f|_A^{-1}(B) \cup f|_U^{-1}(B) = (C\cap A) \cup (D\cap U) \in \overline{\mathcal{B}}^\mathcal{L}.
    $$
    Thus, we can conclude that $f$ is Lebesgue measurable.
\end{solution}

\begin{solution}[5.21]{Beerens, L.}
    Let $f:\mathbb{R}\rightarrow\mathbb{R}$ be a differentiable function. Then for all $a\in\mathbb{R}$, the limit 
    $$
        \lim_{h\rightarrow 0}\frac{f(a+h)-f(a)}{h}
    $$
    exists. This limit is $f'(a)$. Therefore, 
    $$
        \lim_{n\rightarrow\infty}n\left(f\left(a+\frac{1}{n}\right)-f(a)\right) = f'(a).
    $$
    For all $n\in\mathbb{N}$, let 
    $$
        f_n(x) = n\left(f\left(x+\frac{1}{n}\right)-f(x)\right).
    $$
    Then for all $a\in\mathbb{R}$ we have $\lim_{n\rightarrow\infty}f_n(a) = f'(a)$. Therefore $(f_n)$ converges to $f'$ pointwise and we can denote $f'$ by 
    $$
        f' = \limsup_{n\rightarrow\infty}f_n.
    $$
    Note that $f(x+1/n)$ is just a shifted version of $f$ and is therefore measurable. It is clear that $f_n$ is simply a linear combination of two measurable functions and hence is measurable itself. Thus, $(f_n)$ is a sequence of measurable functions and we can conclude that $f'$ is measurable.
\end{solution}