\begin{solution}[2.4]{Castella, A.} 
    A $\sigma$-algebra $\mathcal{G}$ is defined by as a set of subsets such that
    \begin{itemize}
        \item $E \in \mathcal{G}$,
        \item for all infinite sequences $\{A_i\}$ in $\mathcal{G}$ the union $\bigcup_{i=1}^\infty A_i$ also belongs to the set $\mathcal{G}$,
        \item for all infinite sequences $\{A_i\}$ in $\mathcal{G}$ the intersection $\bigcap_{i=1}^\infty A_i$ also belongs to the set $\mathcal{G}$,
        \item for all sets $A$ in $\mathcal{G}$, the set $A^c$ belongs to $\mathcal{G}$.
    \end{itemize}
    Therefore, we begin by verifying that $\Omega \in \mathcal{F}$. From the definition of $f$ it directly follows that
    $$
        f^{-1}(E) = \Omega.
    $$
    Since $\mathcal{G}$ is a $\sigma$-algebra, we find that $E \in \mathcal{G}$. Therefore, we conclude that $\Omega \in \mathcal{F}$.
    
    We now verify the second condition. Let us take an arbitrary infinite sequence $\{A_i\}$ in $\mathcal{F}$. By the definition of $\mathcal{F}$ we know that for all $A_i$, there exists $B_i \in \mathcal{G}$ such that $A_i = f^{-1}(B_i)$. In order to prove the condition, we notice that
    $$
        \bigcup_{i=1}^\infty B_i \in \mathcal{G}.
    $$
    
    Before we continue with the proof, we need to prove an intermediary. Let us take some family of sets $\{C_\alpha\}_{\alpha\in\mathcal{I}}$ where $\mathcal{I}$ is an index set. We will prove that for some function $g : \Omega \rightarrow E$ we have
    $$
        g^{-1}\left(\bigcup_{\alpha\in\mathcal{I}}C_\alpha\right) = \bigcup_{\alpha\in\mathcal{I}}g^{-1}(C_\alpha).
    $$
    We begin by showing that $g^{-1}\left(\bigcup_{\alpha\in\mathcal{I}}C_\alpha\right) \subset \bigcup_{\alpha\in\mathcal{I}}g^{-1}(C_\alpha)$. Let us take an arbitrary $a \in g^{-1}\left(\bigcup_{\alpha\in\mathcal{I}}C_\alpha\right)$. Then we find that there exists some $b \in \bigcup_{\alpha\in\mathcal{I}}C_\alpha$ such that $g(a) = b$. This implies that there exists an $\alpha \in \mathcal{I}$ such that $b \in C_\alpha$. From this we find that
    $$
        a \in g^{-1}(C_\alpha) \subset \bigcup_{\alpha\in\mathcal{I}}g^{-1}(C_\alpha).
    $$
    Since $a$ was chosen arbitrarily, we can conclude that
    $$
        g^{-1}\left(\bigcup_{\alpha\in\mathcal{I}}C_\alpha\right) \subset \bigcup_{\alpha\in\mathcal{I}}g^{-1}(C_\alpha).
    $$
    Now we show that the converse is also true. Let us take an arbitrary $a' \in \bigcup_{\alpha\in\mathcal{I}}g^{-1}(C_\alpha)$. Then we find that there must exist some $\alpha' \in \mathcal{I}$ such that $a' \in g^{-1}(C_{\alpha'})$. We can now choose some $b' \in C_{\alpha'}$ such that $g(a') = b'$. It is clear that $b' \in \bigcup_{\alpha\in\mathcal{I}}C_\alpha$ as well. Thus we know that
    $$
        a' \in g^{-1}\left(\bigcup_{\alpha\in\mathcal{I}}C_\alpha\right).
    $$
    Since again, our choice of $a'$ was arbitrary, we can conclude that
    $$
        \bigcup_{\alpha\in\mathcal{I}}g^{-1}(C_\alpha) \subset g^{-1}\left(\bigcup_{\alpha\in\mathcal{I}}C_\alpha\right).
    $$
    These two inequalities clearly imply that
    $$
        g^{-1}\left(\bigcup_{\alpha\in\mathcal{I}}C_\alpha\right) = \bigcup_{\alpha\in\mathcal{I}}g^{-1}(C_\alpha).
    $$
    
    Using the result that we have just proven we can continue with the proof. For the infinite sequence $\{A_i\}$ we find that
    $$
        f^{-1}\left(\bigcup_{i=1}^\infty B_i\right) = \bigcup_{i=1}^\infty f^{-1}(B_i) = \bigcup_{i=1}^\infty A_i.
    $$
    By the definition of our set $\mathcal{F}$ we now find that
    $$
        \bigcup_{i=1}^\infty A_i \in \mathcal{F}.
    $$
    Therefore, we conclude that the second condition holds for $\mathcal{F}$ as well.
    
    The third condition follows directly from the second and fourth conditions. Therefore, we will first prove the fourth condition. Let us take $A \in \mathcal{F}$. By the definition of $\mathcal{F}$ we find that there exists $B \in \mathcal{G}$ such that
    $$
        A = f^{-1}(B).
    $$
    By the definition of a $\sigma$-algebra we know that $B^c \in \mathcal{G}$. Additionally, we know that
    $$
        f^{-1}(B^c) = f^{-1}(B)^c = A^c.
    $$
    By using these two facts and the definition of the set $\mathcal{F}$ again, we find that
    $$
        A^c \in \mathcal{F}.
    $$
    This proves that the fourth condition holds as well.
    
    Finally, we show that the third condition does indeed follow from the second and fourth condition. Let us take the same infinite sequence $\{A_i\}$ in $\mathcal{F}$ as before. From condition four, we know that the infinite sequence $\{(A_i)^c\}$ is in $\mathcal{F}$ as well. From condition two, we know that
    $$
        \bigcup_{i=1}^\infty (A_i)^c \in \mathcal{F}.
    $$
    Applying condition four again, we find that
    $$
        \left(\bigcup_{i=1}^\infty (A_i)^c\right)^c \in \mathcal{F}.
    $$
    Using standard set theory, we find that
    $$
        \left(\bigcup_{i=1}^\infty (A_i)^c\right)^c = \bigcap_{i=1}^\infty \left((A_i)^c\right)^c = \bigcap_{i=1}^\infty A_i.
    $$
    We can use this to conclude that
    $$
        \bigcap_{i=1}^\infty A_i \in \mathcal{F}.
    $$
    Since our choice was arbitrary, we find that condition three holds for all infinite sequences in $\mathcal{F}$.
    
    Since all of the required conditions hold, we can come to the final conclusion that $\mathcal{F}$ is indeed a $\sigma$-algebra.
\end{solution}

\begin{solution}[2.5]{Castella, A.}
\begin{itemize}
    \item Before we begin with the proof, we will prove the intermediary that if $A,B \in \mathcal{F}$, then $A \setminus B \in \mathcal{F}$. We begin by noting that
    $$
        A \setminus B = A \cap (B^c).
    $$
    From this it becomes very clear that the statement is true. We know by the definition of a $\sigma$-algebra that $B \in \mathcal{F}$ implies $B^c \in \mathcal{F}$. Additionally, we know that intersections of infinite sequences also belong to the same $\sigma$-algebra. We take the sequence $A_1 = A$, $A_i = B^c$ for $i \in \mathbb{N}\setminus\{1\}$. With this we find that $A \cap B^c \in \mathcal{F}$ and therefore
    $$
        A\setminus B \in \mathcal{F}.
    $$
    We now proceed to showing that for all infinite sequences $(A_n)_{n\in\mathbb{N}}$, there exits a mutually disjoint sequence who's union is equal to $\cup_{n\in\mathbb{N}}A_n$. We define the sequence $(E_n)_{n\in\mathbb{N}}$ such that
    $$
        E_n = A_n \setminus \bigcup_{i=1}^{n-1}A_i.
    $$
    By the intermediary and since infinite unions, and by the same argument as in the intermediary, also finite unions are contained in the $\sigma$-algebra, it is easy to see that
    $$
        E_n \in \mathcal{F}
    $$
    for all $n\in\mathbb{N}$. It is clear from the definition of the sequence that it is mutually disjoint and that its union is equal to the union of $(A_n)_{n\in\mathbb{N}}$.
    \item We take the same arbitrary sequence $(A_n)_{n\in\mathbb{N}}$ as in the previous item. We define the sequence $(F_n)_{n\in\mathbb{N}}$ by
    $$
        F_n = A_n \cup \left(\bigcup_{i=1}^{n-1}A_i\right) = \bigcup_{i=1}^n A_i.
    $$
    We first note that it is clear from the definition that this is an increasing sequence, as each element is the union of $A_n$ and all of its predecessors. As mentioned in the previous item, infinite unions are contained in the $\sigma$-algebra as well as finite unions. Therefore we know that
    $$
        F_n \in \mathcal{F}
    $$
    for all $n \in \mathbb{N}$. From the definition of a union we also know that
    $$
        \bigcup_{i=1}^\infty \left(\bigcup_{j=1}^i A_j\right) = \bigcup_{i=1}^\infty A_n.
    $$
    Thus we have now proven that the sequence $(F_n)_{n\in\mathbb{N}}$ is such that $F_n \subset F_{n+1}$ and $\cup_{n\in\mathbb{N}}A_n = \cup_{n\in\mathbb{N}}F_n$.
\end{itemize}
\end{solution}