% \begin{solution}[2.1]{Beerens, L.}
%     Let $(\Omega, \mathcal{A}_0)$ be a measurable space, where $\Omega\neq\varnothing$. Assume that $\mathcal{A}_0$ has an odd number of elements. Choose $n\in\mathbb{N}$ such that $|\mathcal{A}_0|=2n+1$. Choose some set $A\in \mathcal{A}_0$. Since $\mathcal{A}_0$ is a $\sigma$-algebra, we know that $A^c\in\mathcal{A}_0$. Let $\mathcal{A}_1 = \mathcal{A}\setminus\{A, A^c\}$.
%     For all elements $A\in\mathcal{A}_1$, the complement is clearly still in $\mathcal{A}_1$.
%     We can repeat the process by noticing that
%     $$
%         \forall_{A\in\mathcal{A}_k}: A^c\in\mathcal{A}_k,
%     $$
%     for all $k\leq n$.
%     For $k=0$, it is true since we know that $\mathcal{A}_0$ is a $\sigma$-algebra. Assume that for some $k\in\mathbb{N}\cap[0,n)$ we have
%     $$
%         \forall A\in\mathcal{A}_k}: A^c\in\mathcal{A}_k.
%     $$
%     Note that $|\mathcal{A}_k|=2(n-k)+1\geq 3$, since $k\leq n -1$. We can thus create $\mathcal{A}_{k+1}$ using the method of picking some set $A\in\mathcal{
%     A}_{k}$ and defining $\mathcal{A}_{k+1} = \mathcal{A}_k\setminus\{A, A^c\}$. Let $B\in\mathcal{A}_{k+1}\subset\mathcal{A}_0$. By the fact that $\mathcal{A}_0$ is a $\sigma$-algebra, we know that $B^c\in\mathcal{A_0}$. Clearly, $B^c$ has not been removed in any of the $k+1$ steps, since that would imply that $B$ was removed too. Therefore, $B^c\in\mathcal{A}_{k+1}$. By induction, we find that
%     $$
%         \forall_{A\in\mathcal{A}_n}: A^c\in\mathcal{A}_n.
%     $$
%     However, $|\mathcal{A}_n|=1$. Let $A\in\mathcal{A}_n$. Since we assumed that $\Omega\neq \varnothing$, we know that $A^c\neq A$, which leads to a contradiction. Therefore, we can conclude that $\mathcal{A}_0$ has an even amount of elements.
% \end{solution}

\begin{solution}[2.2]{Castella, A.}
    By the definition of a $\sigma$-algebra, all countable unions of sets in $\mathcal{F}$ are also in $\mathcal{F}$. Thus let create a sequence $(A_n)_{n\in\mathbb{N}}$ such that $A_1 = A$ and $A_i = B$ for all $i > 1$. Clearly
    $$
        \bigcup_{i=1}^\infty A_i = A\cup B \in \mathcal{F}.
    $$
    Since $A$ and $B$ are arbitrary, this must hold for all pairs of sets in $\mathcal{F}$. From the fact that $\mathcal{F}$ is a $\sigma$-algebra we find that $A^c,B^c \in \mathcal{F}$. As we have proven that unions of pairs are in $\mathcal{F}$, we find that
    $$
        (A^c \cup B^c)^c = A\cap B \in \mathcal{F}.
    $$
    This again holds for all possible pairs of sets in $\mathcal{F}$. Now let us note that
    $$
        A\setminus B = A\cap B^c.
    $$
    It is clear from the properties that we have already proven and the complementation property of $\sigma$-algebras, we can conclude that
    $$
        A\setminus B \in \mathcal{F}.
    $$
    The final set, $A\Delta B$ follows by definition. The set is defined by $(A\cup B)\setminus (A\cap B)$. It is clear that this is a composition of the properties we have already proven. Since the sets $A$ and $B$ were arbitrary, we can conclude that
    $$
        A\Delta B \in \mathcal{F}.
    $$
\end{solution}

\begin{solution}[2.3]{Castella, A.}
    \begin{itemize}
        \item Using the definition and standard set theory, we find that
        $$
            (\limsup_{n\rightarrow\infty}A_n)^c = \left(\bigcap_{n\in\mathbb{N}}\bigcup_{m\geq n}A_m\right)^c = \bigcup_{n\in\mathbb{N}}\left(\bigcup_{m\geq n}A_m\right)^c = \bigcup_{n\in\mathbb{N}}\bigcap_{m\geq n}(A_m)^c = \liminf_{n\rightarrow\infty}A_n^c.
        $$
        Therefore, the condition holds.
        \item We know that $\mathcal{F}$ is a $\sigma$-algebra, and therefore countable unions and complements are contained in the set. We show that countable intersections follow from those two conditions. By complementation we know that $(A_n)^c$ is in $\mathcal{F}$. Using both conditions in combination we find that
        $$
            \left(\bigcup_{n\in\mathbb{N}}(A_n)^c\right)^c = \bigcap_{n\in\mathbb{N}}A_n \in \mathcal{F}.
        $$
        Thus we find that for all $n \in \mathbb{N}$, we find that
        $$
            \bigcap_{m\geq n}A_m \in \mathcal{F}.
        $$
        Additionally, all countable unions are contained in the set. therefore we find that
        $$
            \bigcup_{n\in\mathbb{N}}\bigcap_{m\geq n}A_m = \liminf_{n\rightarrow\infty}A_n \in \mathcal{F}.
        $$
        Since the sequence is arbitrary, we find that also 
        $$
            \liminf_{n\rightarrow\infty}(A_n)^c \in \mathcal{F}.
        $$
        By item one we find that
        $$
            (\limsup_{n\rightarrow\infty}A_n)^c \in \mathcal{F}.
        $$
        Finally, by the complement property of $\mathcal{F}$ we find that
        $$
            \limsup_{n\rightarrow\infty}A_n \in \mathcal{F}.
        $$
        \item Let us take an arbitrary $a \in \liminf_{n\rightarrow\infty}A_n$. Then we find that there exists $n \in \mathbb{N}$ such that
        $$
            a\in \bigcap_{m\geq n}A_m.
        $$
        Thus we find that for all $m \geq n$, we have
        $$
            a\in A_m.
        $$
        For all $k \in \mathbb{N}$ we can choose $m' \geq \max(k,n)$ such that
        $$
            a\in A_{m'} \subset \bigcup_{m\geq k}A_m.
        $$
        Therefore we find that
        $$
            a\in \bigcap_{n\in\mathbb{N}}\bigcup_{m\geq n}A_m = \limsup_{n\rightarrow\infty}A_n.
        $$
        Since $a$ was arbitrary, we now conclude that
        $$
            \liminf_{n\rightarrow\infty}A_n \subset \limsup_{n\rightarrow\infty}A_n.
        $$
        \item Let us take some $a \in \limsup_{n\rightarrow\infty}A_n$. By the definition of intersections we know that
        $$
            a \in \bigcup_{i=1}^\infty A_m
        $$
        Therefore there exists $m_0 \in \mathbb{N}$ such that
        $$
            a\in A_{m_0}.
        $$
        Again, by the definition of intersections, we know that for all $n \in \mathbb{N}$ we have
        $$
            a\in \bigcup_{m\geq n}A_m.
        $$
        Therefore, let us choose $n' \geq m_0$, then we know that
        $$
            a\in \bigcup_{m\geq n'}A_m
        $$
        which implies that there exists $m_1 \in \mathbb{N}$ such that
        $$
            a\in A_{m_1}.
        $$
        We find that using a simple induction we can create an increasing sequence $(m_n)_{n\in\mathbb{N}}$ such that for all $n$, we have
        $$
            a\in A_{m_n}.
        $$
        Thus we conclude that
        $$
            a\in \{\omega\in\Omega:\omega\in A_n\text{ for infinitely many }n\}.
        $$
        Let us now take an arbitrary $b \in \{\omega\in\Omega:\omega\in A_n\text{ for infinitely many }n\}$. Clearly, by the definition of the set we can choose a strictly increasing sequence $(n_i)_{i\in\mathbb{N}}$ such that
        $$
            b\in A_{n_i}
        $$
        for all $i$. Since for all $i$, we know that $n_i$ is integer, it is clear that
        $$
            \lim_{i\rightarrow\infty}n_i = \infty.
        $$
        Let us take an arbitrary $k\in\mathbb{N}$. Using the limit we find that there exists $i$ such that $n_i > k$ and therefore
        $$
            b \in A_{n_i} \subset \bigcup_{m\geq k}A_m.
        $$
        Since $k$ was arbitrary, this must hold for all $k$. Thus we find that
        $$
            b \in \bigcap_{n\in\mathbb{N}}\bigcup_{m\geq n}A_m = \limsup_{n\rightarrow\infty}A_n.
        $$
        Therefore, we now find that
        $$
            \{\omega\in\Omega:\omega\in A_n\text{ for infinitely many }n\} \subset \limsup_{n\rightarrow\infty}A_n
        $$
        and we can finally conclude that
        $$
            \limsup_{n\rightarrow\infty}A_n = \{\omega\in\Omega:\omega\in A_n\text{ for infinitely many }n\}.
        $$
        \item Let us take an arbitrary $a \in \liminf_{n\rightarrow\infty}A_n$. By the definition of a union, we find that there exists $n\in\mathbb{N}$ such that
        $$
            a \in \bigcap_{m\geq n}A_m.
        $$
        By the definition of intersections, we find that for all $m\geq n$ we have
        $$
            a \in A_m.
        $$
        Clearly this implies that
        $$
            a\in \{\omega\in\Omega:\exists_{m\in\mathbb{N}}\text{ such that }\omega\in A_n\text{ for all }n\geq m\}.
        $$
        Since $a$ was chosen arbitrarily we find that
        $$
            \liminf_{n\rightarrow\infty}A_n \subset \{\omega\in\Omega:\exists_{m\in\mathbb{N}}\text{ such that }\omega\in A_n\text{ for all }n\geq m\}.
        $$
        Now let us choose an arbitrary $b \in \{\omega\in\Omega:\exists_{m\in\mathbb{N}}\text{ such that }\omega\in A_n\text{ for all }n\geq m\}$. By the definition, there exists $n' \in \mathbb{N}$ such that
        $$
            b \in A_n
        $$
        for all $n \geq n'$. Therefore we find that
        $$
            b \in \bigcap_{m\geq n'}A_m.
        $$
        By the definition of a union we find that
        $$
            b\in \bigcap_{m\geq n'}A_m \subset \bigcup_{n\in\mathbb{N}}\bigcap_{m\geq n}A_m = \liminf_{n\rightarrow\infty}A_n.
        $$
        Therefore we find that
        $$
            \{\omega\in\Omega:\exists_{m\in\mathbb{N}}\text{ such that }\omega\in A_n\text{ for all }n\geq m\} \subset \liminf_{n\rightarrow\infty}A_n.
        $$
        We can now conclude that
        $$
            \liminf_{n\rightarrow\infty}A_n = \{\omega\in\Omega:\exists_{m\in\mathbb{N}}\text{ such that }\omega\in A_n\text{ for all }n\geq m\}.
        $$
    \end{itemize}
\end{solution}

\begin{solution}[2.4]{Castella, A.}
    A $\sigma$-algebra $\mathcal{G}$ is defined as a set of subsets such that
    \begin{itemize}
        \item $E \in \mathcal{G}$,
        \item for all infinite sequences $\{A_i\}$ in $\mathcal{G}$ the union $\bigcup_{i=1}^\infty A_i$ also belongs to the set $\mathcal{G}$,
        \item for all sets $A$ in $\mathcal{G}$, the set $A^c$ belongs to $\mathcal{G}$.
    \end{itemize}
    Therefore, we begin by verifying that $\Omega \in \mathcal{F}$. From the definition of $f$ it directly follows that
    $$
        f^{-1}(E) = \Omega.
    $$
    Since $\mathcal{G}$ is a $\sigma$-algebra, we find that $E \in \mathcal{G}$. Therefore, we conclude that $\Omega \in \mathcal{F}$.

    We now verify the second condition. Let us take an arbitrary infinite sequence $\{A_i\}$ in $\mathcal{F}$. By the definition of $\mathcal{F}$ we know that for all $A_i$, there exists $B_i \in \mathcal{G}$ such that $A_i = f^{-1}(B_i)$. In order to prove the condition, we notice that
    $$
        \bigcup_{i=1}^\infty B_i \in \mathcal{G}.
    $$

    Before we continue with the proof, we need to prove an intermediary. Let us take some family of sets $\{C_\alpha\}_{\alpha\in\mathcal{I}}$ where $\mathcal{I}$ is an index set. We will prove that for some function $g : \Omega \rightarrow E$ we have
    $$
        g^{-1}\left(\bigcup_{\alpha\in\mathcal{I}}C_\alpha\right) = \bigcup_{\alpha\in\mathcal{I}}g^{-1}(C_\alpha).
    $$
    We begin by showing that $g^{-1}\left(\bigcup_{\alpha\in\mathcal{I}}C_\alpha\right) \subset \bigcup_{\alpha\in\mathcal{I}}g^{-1}(C_\alpha)$. Let us take an arbitrary $a \in g^{-1}\left(\bigcup_{\alpha\in\mathcal{I}}C_\alpha\right)$. Then we find that there exists some $b \in \bigcup_{\alpha\in\mathcal{I}}C_\alpha$ such that $g(a) = b$. This implies that there exists an $\alpha \in \mathcal{I}$ such that $b \in C_\alpha$. From this we find that
    $$
        a \in g^{-1}(C_\alpha) \subset \bigcup_{\alpha\in\mathcal{I}}g^{-1}(C_\alpha).
    $$
    Since $a$ was chosen arbitrarily, we can conclude that
    $$
        g^{-1}\left(\bigcup_{\alpha\in\mathcal{I}}C_\alpha\right) \subset \bigcup_{\alpha\in\mathcal{I}}g^{-1}(C_\alpha).
    $$
    Now we show that the converse is also true. Let us take an arbitrary $a' \in \bigcup_{\alpha\in\mathcal{I}}g^{-1}(C_\alpha)$. Then we find that there must exist some $\alpha' \in \mathcal{I}$ such that $a' \in g^{-1}(C_{\alpha'})$. We can now choose some $b' \in C_{\alpha'}$ such that $g(a') = b'$. It is clear that $b' \in \bigcup_{\alpha\in\mathcal{I}}C_\alpha$ as well. Thus we know that
    $$
        a' \in g^{-1}\left(\bigcup_{\alpha\in\mathcal{I}}C_\alpha\right).
    $$
    Since again, our choice of $a'$ was arbitrary, we can conclude that
    $$
        \bigcup_{\alpha\in\mathcal{I}}g^{-1}(C_\alpha) \subset g^{-1}\left(\bigcup_{\alpha\in\mathcal{I}}C_\alpha\right).
    $$
    These two inequalities clearly imply that
    $$
        g^{-1}\left(\bigcup_{\alpha\in\mathcal{I}}C_\alpha\right) = \bigcup_{\alpha\in\mathcal{I}}g^{-1}(C_\alpha).
    $$

    Using the result that we have just proven we can continue with the proof. For the infinite sequence $\{A_i\}$ we find that
    $$
        f^{-1}\left(\bigcup_{i=1}^\infty B_i\right) = \bigcup_{i=1}^\infty f^{-1}(B_i) = \bigcup_{i=1}^\infty A_i.
    $$
    By the definition of our set $\mathcal{F}$ we now find that
    $$
        \bigcup_{i=1}^\infty A_i \in \mathcal{F}.
    $$
    Therefore, we conclude that the second condition holds for $\mathcal{F}$ as well.

    We now prove the third and final condition. Let us take $A \in \mathcal{F}$. By the definition of $\mathcal{F}$ we find that there exists $B \in \mathcal{G}$ such that
    $$
        A = f^{-1}(B).
    $$
    By the definition of a $\sigma$-algebra we know that $B^c \in \mathcal{G}$. Additionally, we know that
    $$
        f^{-1}(B^c) = f^{-1}(B)^c = A^c.
    $$
    By using these two facts and the definition of the set $\mathcal{F}$ again, we find that
    $$
        A^c \in \mathcal{F}.
    $$
    This proves that the third condition holds as well.

    Since all of the required conditions hold, we can come to the final conclusion that $\mathcal{F}$ is indeed a $\sigma$-algebra.
\end{solution}

\begin{solution}[2.5]{Castella, A.}
\begin{itemize}
    \item Before we begin with the proof, we will prove the intermediary that if $A,B \in \mathcal{F}$, then $A \setminus B \in \mathcal{F}$. We begin by noting that
    $$
        A \setminus B = A \cap (B^c).
    $$
    From this it becomes very clear that the statement is true. We know by the definition of a $\sigma$-algebra that $B \in \mathcal{F}$ implies $B^c \in \mathcal{F}$. Additionally, we know that intersections of infinite sequences also belong to the same $\sigma$-algebra. We take the sequence $A_1 = A$, $A_i = B^c$ for $i \in \mathbb{N}\setminus\{1\}$. With this we find that $A \cap B^c \in \mathcal{F}$ and therefore
    $$
        A\setminus B \in \mathcal{F}.
    $$
    We now proceed to showing that for all infinite sequences $(A_n)_{n\in\mathbb{N}}$, there exits a mutually disjoint sequence who's union is equal to $\cup_{n\in\mathbb{N}}A_n$. We define the sequence $(E_n)_{n\in\mathbb{N}}$ such that
    $$
        E_n = A_n \setminus \bigcup_{i=1}^{n-1}A_i.
    $$
    By the intermediary and since infinite unions, and by the same argument as in the intermediary, also finite unions are contained in the $\sigma$-algebra, it is easy to see that
    $$
        E_n \in \mathcal{F}
    $$
    for all $n\in\mathbb{N}$. It is clear from the definition of the sequence that it is mutually disjoint and that its union is equal to the union of $(A_n)_{n\in\mathbb{N}}$.
    \item We take the same arbitrary sequence $(A_n)_{n\in\mathbb{N}}$ as in the previous item. We define the sequence $(F_n)_{n\in\mathbb{N}}$ by
    $$
        F_n = A_n \cup \left(\bigcup_{i=1}^{n-1}A_i\right) = \bigcup_{i=1}^n A_i.
    $$
    We first note that it is clear from the definition that this is an increasing sequence, as each element is the union of $A_n$ and all of its predecessors. As mentioned in the previous item, infinite unions are contained in the $\sigma$-algebra as well as finite unions. Therefore we know that
    $$
        F_n \in \mathcal{F}
    $$
    for all $n \in \mathbb{N}$. From the definition of a union we also know that
    $$
        \bigcup_{i=1}^\infty \left(\bigcup_{j=1}^i A_j\right) = \bigcup_{i=1}^\infty A_n.
    $$
    Thus we have now proven that the sequence $(F_n)_{n\in\mathbb{N}}$ is such that $F_n \subset F_{n+1}$ and $\cup_{n\in\mathbb{N}}A_n = \cup_{n\in\mathbb{N}}F_n$.
\end{itemize}
\end{solution}

\begin{solution}[2.13]{Beerens, L.}
    Let $\Omega_1$ and $\Omega_2$ be two non-empty sets, and let $\mathcal{F}_1$ and $\mathcal{F}_2$ be $\sigma$-algebras on $\Omega_1$ and $\Omega_2$ respectively. We consider the product $\sigma$-algebra on $\Omega_1\times\Omega_2$ given by
    $$
        \mathcal{F}_1\otimes\mathcal{F}_2:=\sigma(\{ A_1\times A_2: A_1\in\mathcal{F}_1, A_2\in\mathcal{F}_2 \}).
    $$
    Suppose that $\mathcal{F}_1$ is generated by $\mathcal{A}_1$ and $\mathcal{F}_2$ is generated by $\mathcal{A}_2$. Let $\mathcal{F}:=\mathcal{F}_1\otimes\mathcal{F}_2$ and
    $$
        \mathcal{S}:=\sigma(\{ A_1\times A_2: A_1\in\mathcal{A}_1, A_2\in\mathcal{A}_2 \}).
    $$
    Let
    $$
        A\in\{ A_1\times A_2: A_1\in\mathcal{A}_1, A_2\in\mathcal{A}_2 \}.
    $$
    Then $A=A_1\times A_2$ for some $A_1\in\mathcal{A}_1$ and $A_2\in\mathcal{A}_2$. Since $\mathcal{A}_1\subset\mathcal{F}_1$ and $\mathcal{A}_2\subset\mathcal{F}_2$, we find that
    $$
        A\in\{ A_1\times A_2: A_1\in\mathcal{F}_1, A_2\in\mathcal{F}_2 \}.
    $$
    Therefore,
    $$
        \{ A_1\times A_2: A_1\in\mathcal{A}_1, A_2\in\mathcal{A}_2 \}\subset\{ A_1\times A_2: A_1\in\mathcal{F}_1, A_2\in\mathcal{F}_2 \},
    $$
    from which it follows that $\mathcal{S}\subset\mathcal{F}$ (By Problem 2.12).

    To prove the converse, suppose that $F\in\{ A_1\times A_2: A_1\in\mathcal{F}_1, A_2\in\mathcal{F}_2 \}$. Then there exist $A_1\in\mathcal{F}_1$ and $A_2\in\mathcal{F}_2$ such that $F = A_1\times A_2 = (A_1\times\Omega_2)\cap(\Omega_1\times A_2)$. By definition for all $A\in\mathcal{A}_1$ we have
    $$
         A\times\Omega_2\in \mathcal{S}.
    $$
    Therefore, the collection $\{A \in \cF_1: A\times \Omega_2 \in \cS\}$ is a $\sigma$-algebra containing $\cA_1$ and contained in $\cF_1$. It follows that, 
    $$
    \cF_1 = \sigma(\mathcal{A}_1) = \{A \in \cF_1: A\times \Omega_2 \in \cS\}.
    $$
    Since $A_1\in\mathcal{F}_1$, it now follows that $A_1\times\Omega_2\in \mathcal{S}$. Analogously, $\Omega_1\times A_2\in \mathcal{S}$. Since $\mathcal{S}$ is a $\sigma$-algebra, it follows that $F = (A_1\times\Omega_2)\cap(\Omega_1\times A_2) \in \mathcal{S}$. Thus, $\mathcal{F}\subset\mathcal{S}$, from which we can conclude that $\mathcal{F}=\mathcal{S}$, as was to be shown.
\end{solution}

\begin{solution}[2.14]{Castella, A.}
    In order to prove equality of the sets, we will prove that they are both subsets of one another.
    \begin{itemize}
        \item As we know, the set $\mathcal{B}_{\mathbb{R}^2}$ is generated by the $\pi$-system of open rectangles. Let us assume that $\mathcal{A}$ is the set of open rectangles in $\mathbb{R}^2$. Let us take some arbitrary $A \in \mathcal{A}$, then there exists $a,b,c,d \in \mathbb{R}$ such that $a < b$, $c < d$, and $A = (a,b)\times(c,d)$. Since the Borel set $\mathcal{B}_\mathbb{R}$ is generated by the set of open intervals in $\mathbb{R}$, we find that $(a,b), (c,d) \in \mathcal{B}_\mathbb{R}$. From this we immediately find that $(a,b)\times(c,d) \in \mathcal{B}_\mathbb{R}\times\mathcal{B}_\mathbb{R}$. Since $\mathcal{B}_\mathbb{R}\otimes\mathcal{B}_\mathbb{R}$ is the $\sigma$-algebra generated by the cross product $\mathcal{B}_\mathbb{R}\times\mathcal{B}_\mathbb{R}$ we find that
        $$
            \mathcal{B}_{\mathbb{R}^2} \subset \mathcal{B}_\mathbb{R}\otimes\mathcal{B}_\mathbb{R}.
        $$
        \item We now prove the converse. Let us begin by defining the projections $\pi_1$ and $\pi_2$ as
        $$
            \pi_1(x,y) = x
        $$
        and
        $$
            \pi_2(x,y) = y.
        $$
        We will show that these two functions are measurable with respect to $\cB_{\bbR^2}$ and $\cB_{\bbR}$. Let us take an arbitrary $t \in \mathbb{R}$. We will show that the set
        $$
            A = \{(x\times y) \in \mathbb{R}^2 : \pi_1(x,y) < t\}
        $$
        is a Borel set. We find that $\pi_1(x,y) < t$ if and only if $x < t$. Thus we find that
        $$
            A = (-\infty, t) \times \mathbb{R}.
        $$
        Let us define the set $B_n$ as
        $$
            B_n = (-n,t) \times (-n,n)
        $$
        Clearly, for all $n\in\mathbb{N}$, the set $B_n$ is an open rectangle and therefore $B_n \in \cB_{\bbR^2}$. We also find that
        $$
            \bigcup_{i=1}^\infty B_n = (-\infty,t)\times \mathbb{R} = A.
        $$
        Since $\cB_{\bbR^2}$ is a $\sigma$-algebra, it contains all countable unions of its sets. Therefore, we find that
        $$
            \bigcup_{i=1}^\infty B_n = A \in \cB_{\bbR^2}.
        $$
        Thus the set $A$ is indeed a Borel measurable set. Since $t$ was chosen arbitrarily, this holds for all $t \in \mathbb{R}$. Thus $\pi_1$ is a measurable function. The proof is analogous for $\pi_2$. Since both of these functions are measurable, we know that the preimage of a set in $\cB_{\bbR}$ is in $\cB_{\bbR^2}$. Therefore, for all $A \in \cB_{\bbR}$, we find that
        $$
            A \times \mathbb{R} \in \cB_{\bbR^2},
        $$
        by the measurability of $\pi_1$ and
        $$
            \mathbb{R} \times A \in \cB_{\bbR^2},
        $$
        by the measurability of $\pi_2$. Let us take arbitrary $A,B \in \cB_{\bbR}$. By our previous result and since $\cB_{\bbR^2}$ is a $\sigma$-algebra and thus contains countable intersections, we know that
        $$
            (A\times\mathbb{R})\cap(\mathbb{R}\times B) = A\times B \in \cB_{\bbR^2}.
        $$
        By the fact that $A$ and $B$ were chosen arbitrarily we find that
        $$
            \cB_{\bbR} \times \cB_{\bbR} \subset \cB_{\bbR^2}.
        $$
        Since $\cB_{\bbR}\otimes\cB_{\bbR}$ is the $\sigma$-algebra generated by the Cartesian product of $\cB_{\bbR}$ with itself, we can conclude that
        $$
            \cB_{\bbR}\otimes\cB_{\bbR} \subset \cB_{\bbR^2}.
        $$
    \end{itemize}
    Combining both of the results, we arrive at the final conclusion that
    $$
        \mathcal{B}_{\mathbb{R}^2} = \mathcal{B}_\mathbb{R}\otimes\mathcal{B}_\mathbb{R}.
    $$
\end{solution} 